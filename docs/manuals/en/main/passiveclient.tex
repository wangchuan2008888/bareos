\section{Passive Clients}
\label{PassiveClient}

The normal way of initializing the data channel (the channel where the backup data itself is transported)
is done by the \bareosFd (client) that connects to the \bareosSd.

In many setups, this can cause problems, as this means that:
\begin{itemize}
    \item The client must be able to resolve the name of the \bareosSd (often not true, you have to do tricks with the hosts file)
    \item The client must be allowed to create a new connection.
    \item The client must be able to connect to the \bareosSd over the network (often difficult over NAT or Firewall)
\end{itemize}

By using Passive Client, the initialization of the datachannel is reversed, so that the storage daemon connects to the \bareosFd.
This solves almost every problem created by firewalls, NAT-gateways and resolving issues, as
\begin{itemize}
    \item The \bareosSd initiates the connection, and thus can pass through the same or similar firewall rules that the director already has to access the \bareosFd.
    \item The client never initiates any connection, thus can be completely firewalled.
    \item The client never needs any name resolution and is totally independent from any resolving issues.
\end{itemize}

\begin{center}
\includegraphics[width=0.60\textwidth]{\idir passive-client-communication}
\end{center}

\subsection{Usage}

To use this new feature, just configure  \linkResourceDirective{Dir}{Client}{Passive}=yes in the client definition of the \bareosDir:
\begin{bconfig}{Enable passive mode in bareos-dir.conf}
Client {
   Name = client1-fd
   Password = "secretpassword"
   <input>Passive = yes</input>
   [...]
}
\end{bconfig}

Also, prior to bareos version 15, you need to set \linkResourceDirective{Fd}{Client}{Compatible}=no in the \file{bareos-fd.conf} configuration file.
Since Bareos Version 15, the compatible option is set to no per default and does not need to be specified anymore.
\begin{bconfig}{Disable compatible mode for the \bareosFd in bareos-fd.conf} 
Director {
  Name = bareos-dir
  Password = "secretpassword"
}

Client {
   Name = client1-fd
   [...]
   <input>Compatible = no</input>
}
\end{bconfig}
