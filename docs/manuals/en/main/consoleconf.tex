%%
%%

\chapter{Console Configuration}
\label{ConsoleConfChapter}
\index[general]{Configuration!Console}
\index[general]{Console Configuration}

The Console configuration file is the simplest of all the configuration files,
and in general, you should not need to change it except for the password. It
simply contains the information necessary to contact the Director or
Directors.

For a general discussion of the syntax of configuration files and their
resources including the data types recognized by {\bf Bareos}, please see
the \ilink{Configuration}{ConfigureChapter} chapter of this manual.

The following Console Resource definition must be defined:

\section{Director Resource}
\label{ConsoleResourceDirector}
\index[general]{Director Resource}
\index[general]{Resource!Director}

The Director resource defines the attributes of the Director running on the
network. You may have multiple Director resource specifications in a single
Console configuration file. If you have more than one, you will be prompted to
choose one when you start the {\bf Console} program.

\input{autogenerated/bconsole-resource-director-table.tex}
\defDirective{Console}{Director}{Address}{}{}{%
Where the address is a host name,  a fully qualified domain name, or a network
address used to connect  to the Director.
}

\defDirective{Console}{Director}{Description}{}{}{%
}

\defDirective{Console}{Director}{Dir Port}{}{}{%
This port must be identical to the
\linkResourceDirective{Dir}{Director}{Dir Port} specified in the \nameref{DirectorChapter} file.
}

\defDirective{Console}{Director}{Heartbeat Interval}{}{}{%
}

\defDirective{Console}{Director}{Name}{}{}{%
The director name used to select among different Directors, otherwise, this
name is not used.
}

\defDirective{Console}{Director}{Password}{}{}{%
This password is used to authenticate when connecting to the \bareosDir as default console.
It must correspond to \linkResourceDirective{Dir}{Director}{Password}.
}

\defDirective{Console}{Director}{TLS Authenticate}{}{}{%
}

\defDirective{Console}{Director}{TLS CA Certificate Dir}{}{}{%
}

\defDirective{Console}{Director}{TLS CA Certificate File}{}{}{%
}

\defDirective{Console}{Director}{TLS Certificate}{}{}{%
}

\defDirective{Console}{Director}{TLS Certificate Revocation List}{}{}{%
}

\defDirective{Console}{Director}{TLS Enable}{}{}{%
Bareos can be configured to encrypt all its network traffic. See chapter \nameref{TlsDirectives} to see how the Bareos Director (and the other components) have to be configured to use TLS.
}

\defDirective{Console}{Director}{TLS Key}{}{}{%
}

\defDirective{Console}{Director}{TLS Require}{}{}{%
}

\defDirective{Console}{Director}{TLS Verify Peer}{}{}{%
}

\input{autogenerated/bconsole-resource-director-description.tex}

An actual example might be:

\footnotesize
\begin{verbatim}
Director {
  Name = HeadMan
  address = rufus.cats.com
  password = xyz1erploit
}
\end{verbatim}
\normalsize

\hide{
\section{ConsoleFont Resource}
\index[general]{Resource!ConsoleFont}
\index[general]{ConsoleFont Resource}

The ConsoleFont resource is available only in the GNOME version of the
console. It permits you to define the font that you want used to display in
the main listing window.

\begin{description}

\item [ConsoleFont]
\index[console]{ConsoleFont}
Start of the ConsoleFont directives.

\item [Name = {\textless}name{\textgreater}] \hfill \\
\index[console]{Name}
The name of the font.

\item [Font = {\textless}Pango Font Name{\textgreater}] \hfill \\
\index[console]{Font}
The string value given here defines the desired font. It  is specified in the
Pango format. For example, the default specification is:

\footnotesize
\begin{verbatim}
Font = "LucidaTypewriter 9"
\end{verbatim}
\normalsize

\end{description}

Thanks to Phil Stracchino for providing the code for this feature.

An different example might be:

\footnotesize
\begin{verbatim}
ConsoleFont {
  Name = Default
  Font = "Monospace 10"
}
\end{verbatim}
\normalsize
}

\section{Console Resource}
\label{ConsoleResourceConsole}
\index[general]{Console Resource}
\index[general]{Resource!Console}

There are three different kinds of consoles, which the administrator or
user can use to interact with the Director.  These three kinds of consoles
comprise three different security levels.

\begin{itemize}
\item The first console type is an \name{admin} or \name{anonymous} or \name{default}
console, which has full privileges.  There is no console resource
necessary for this type since the password is specified in the Director
resource.  Typically you would use this console
only for administrators.

\item The second type of console is a
"named" or "restricted" console defined within a Console resource in
both the Director's configuration file and in the Console's
configuration file.  Both the names and the passwords in these two
entries must match much as is the case for Client programs.

This second type of console begins with absolutely no privileges except
those explicitly specified in the Director's Console resource.  Note,
the definition of what these restricted consoles can do is determined
by the Director's conf file.

Thus you may define within the Director's conf file multiple Consoles
with different names and passwords, sort of like multiple users, each
with different privileges.  As a default, these consoles can do
absolutely nothing -- no commands what so ever.  You give them
privileges or rather access to commands and resources by specifying
access control lists in the Director's Console resource.  This gives the
administrator fine grained control over what particular consoles (or
users) can do.

\item The third type of console is similar to the above mentioned
restricted console in that it requires a Console resource definition in
both the Director and the Console.  In addition, if the console name,
provided on the {\bf Name =} directive, is the same as a Client name,
the user of that console is permitted to use the {\bf SetIP} command to
change the Address directive in the Director's client resource to the IP
address of the Console.  This permits portables or other machines using
DHCP (non-fixed IP addresses) to "notify" the Director of their current
IP address.
\end{itemize}

The Console resource is optional and need not be specified. However, if it is
specified, you can use ACLs (Access Control Lists) in the Director's
configuration file to restrict the particular console (or user) to see only
information pertaining to his jobs or client machine.

You may specify as many Console resources in the console's conf file. If
you do so, generally the first Console resource will be used.  However, if
you have multiple Director resources (i.e. you want to connect to different
directors), you can bind one of your Console resources to a particular
Director resource, and thus when you choose a particular Director, the
appropriate Console configuration resource will be used. See the "Director"
directive in the Console resource described below for more information.

Note, the Console resource is optional, but can be useful for
restricted consoles as noted above.

\input{autogenerated/bconsole-resource-console-table.tex}
\defDirective{Console}{Console}{Description}{}{}{%
}

\defDirective{Console}{Console}{Director}{}{}{%
If this directive is specified, this Console resource will be
used by bconsole when that particular director is selected
when first starting bconsole.  I.e. it binds a particular console
resource with its name and password to a particular director.
}

\defDirective{Console}{Console}{Heartbeat Interval}{}{}{%
This directive is optional and if specified will cause the Console to
set a keepalive interval (heartbeat) in seconds on each of the sockets
to communicate with the Director.  It is implemented only on systems
(Linux, ...) that provide the {\bf setsockopt} TCP\_KEEPIDLE function.
If the value is set to 0 (zero), no change is made to the socket.
}

\defDirective{Console}{Console}{History File}{}{}{%
If this directive is specified and the console is compiled with readline support,
it will use the given filename as history file.
If not specified, the history file will be named \file{~/.bconsole_history}
}

\defDirective{Console}{Console}{History Length}{}{}{%
If this directive is specified the history file will be truncated after \configdirective{HistoryLength} entries.
}

\defDirective{Console}{Console}{Name}{}{}{%
The Console name used to allow a restricted console to change
its IP address using the SetIP command. The SetIP command must
also be defined in the Director's conf CommandACL list.
}

\defDirective{Console}{Console}{Password}{}{}{%
If this password is supplied, then the password specified in the
Director resource of you Console conf will be ignored.  See below
for more details.
}

\defDirective{Console}{Console}{Rc File}{}{}{%
}

\defDirective{Console}{Console}{TLS Authenticate}{}{}{%
}

\defDirective{Console}{Console}{TLS CA Certificate Dir}{}{}{%
}

\defDirective{Console}{Console}{TLS CA Certificate File}{}{}{%
}

\defDirective{Console}{Console}{TLS Certificate}{}{}{%
}

\defDirective{Console}{Console}{TLS Certificate Revocation List}{}{}{%
}

\defDirective{Console}{Console}{TLS Enable}{}{}{%
Bareos can be configured to encrypt all its network traffic. See chapter \nameref{TlsDirectives} to see how the Bareos Director (and the other components) have to be configured to use TLS.
}

\defDirective{Console}{Console}{TLS Key}{}{}{%
}

\defDirective{Console}{Console}{TLS Require}{}{}{%
}

\defDirective{Console}{Console}{TLS Verify Peer}{}{}{%
}

\input{autogenerated/bconsole-resource-console-description.tex}

\section{Example Console Configuration File}
\index[general]{Configuration!bconsole}

A Console configuration file might look like this:

\begin{bconfig}{bconsole configuration}
Director {
  Name = "bareos.example.com-dir"
  address = "bareos.example.com"
  Password = "PASSWORD"
}
\end{bconfig}

With this configuration, the console program (e.g. \command{bconsole}) will try to connect
to a \bareosDir named \name{bareos.example.com-dir}
at the network address \host{bareos.example.com} and authenticate to the admin console using the password \name{PASSWORD}.

\subsection{Using Named Consoles}
\label{sec:ConsoleAccessExample}
The following configuration files were supplied by Phil Stracchino.

To use named consoles from \command{bconsole}, use a \file{bconsole.conf} configuration file like this:
\begin{bconfig}{bconsole: restricted-user}
Director {
   Name = bareos-dir
   Address = myserver
   Password = "XXXXXXXXXXX"
}

Console {
   Name = restricted-user
   Password = "RUPASSWORD"
}
\end{bconfig}

Where the Password in the Director section is deliberately incorrect and the
Console resource is given a name, in this case \argument{restricted-user}. Then
in the Director configuration (not directly accessible by the user), we define:

\begin{bareosConfigResource}{bareos-dir}{console}{restricted-user}
Console {
  Name = restricted-user
  Password = "RUPASSWORD"
  JobACL = "Restricted Client Save"
  ClientACL = restricted-client
  StorageACL = main-storage
  ScheduleACL = *all*
  PoolACL = *all*
  FileSetACL = "Restricted Client's FileSet"
  CatalogACL = MyCatalog
  CommandACL = run
}
\end{bareosConfigResource}

The user login into the Director from his Console will get logged in as \resourcename{Dir}{Console}{restricted-user}
and he will only be able to see or access a Job with the
name \resourcename*{Dir}{Job}{Restricted Client Save}, a Client with the name \resourcename*{Dir}{Client}{restricted-client},
a storage device \resourcename*{Dir}{Storage}{main-storage}, any Schedule or Pool,
a FileSet named \resourcename*{Dir}{FileSet}{Restricted Client's FileSet}, a Catalog named \resourcename*{Dir}{Catalog}{MyCatalog}
and the only command he can use in the Console is the \bcommand{run}{} command.
In other words, this user is rather limited in what he can see
and do with Bareos.
For details how to configure ACLs, see the \dt{Acl} data type description.

The following is an example of a  \file{bconsole.conf} file that can access
several Directors and has different Consoles depending on the Director:

\begin{bconfig}{bconsole: multiple consoles}
Director {
   Name = bareos-dir
   Address = myserver
   Password = "XXXXXXXXXXX"    # no, really.  this is not obfuscation.
}

Director {
   Name = SecondDirector
   Address = secondserver
   Password = "XXXXXXXXXXX"    # no, really.  this is not obfuscation.
}

Console {
   Name = restricted-user
   Password = "RUPASSWORD"
   Director = MyDirector
}

Console {
   Name = restricted-user2
   Password = "OTHERPASSWORD"
   Director = SecondDirector
}
\end{bconfig}

The second Director referenced at \resourcename{Dir}{Director}{secondserver} might look
like the following:

\begin{bareosConfigResource}{bareos-dir}{console}{restricted-user2}
Console {
  Name = restricted-user2
  Password = "OTHERPASSWORD"
  JobACL = "Restricted Client Save"
  ClientACL = restricted-client
  StorageACL = second-storage
  ScheduleACL = *all*
  PoolACL = *all*
  FileSetACL = "Restricted Client's FileSet"
  CatalogACL = RestrictedCatalog
  CommandACL = run, restore
  WhereACL = "/"
}
\end{bareosConfigResource}
