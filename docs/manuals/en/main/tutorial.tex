%%
%%

\chapter{Tutorial}
\label{TutorialChapter}
\index[general]{Tutorial}

This chapter will guide you through running Bareos. To do so, we assume you
have installed Bareos.
However, we assume that you have not modified the configuration.
The examples in this chapter use the default configuration files
and will write the volumes to disk in your \path|/var/lib/bareos/storage/|
directory.

The general flow of running Bareos is:

\begin{enumerate}
\item Start the Database (if using \postgresql or \mysql)
\item \nameref{InstallChapter}
\item Start the Bareos Daemons
\item Start the Console program to interact with the \bareosDir
\item Run a job
%\item When the Volume fills, unmount the Volume, if it is a  tape, label a new
%   one, and continue running. In this  chapter, we will write only to disk files
%   so you won't  need to worry about tapes for the moment.
\item Test recovering some files from the Volume just written to  ensure the
   backup is good and that you know how to recover.  Better test before disaster
   strikes
\item Add a second client.
\end{enumerate}

Each of these steps is described in more detail below.


\section{Starting the Database}

If you are using \postgresql or \mysql as the Bareos database, you should start
it before you install Bareos.
If you are using \sqlite you need do nothing. \sqlite is automatically
started by the \bareosDir.

\section{Installing Bareos}

For installing Bareos, follow the instructions from the \nameref{InstallChapter} chapter.

\section{Starting the Daemons}
\label{StartDaemon}
\index[general]{Starting the Daemons}
\index[general]{Daemon!Start}

Assuming you have installed the packages,
to start the three daemons, from your installation directory, simply enter:

\begin{bconsole}{start services}
service bareos-dir start
service bareos-sd start
service bareos-fd start
\end{bconsole}

\section{Using the Director to Query and Start Jobs}

\label{sec:TuturialBconsole}

To communicate with the \bareosDir and to query the state of Bareos or run jobs,
the \command{bconsole} program can be used as a textual interface.
Alternatively, for most purposes, also the \ilink{\bareosWebui}{sec:webui} can be used,
but for simplicity, here we will describe only the \command{bconsole} program.

The \command{bconsole} runs the Bareos Console program, which connects to the
\bareosDir. Since Bareos is a network program, you can run the Console
program anywhere on your network. Most frequently, however, one runs it on the
same machine as the \bareosDir. Normally, the Console program will print
something similar to the following:

\begin{commands}{bconsole}
<command>bconsole</command>
Connecting to Director bareos:9101
Enter a period to cancel a command.
*
\end{commands}

The asterisk is the console command prompt.

Type \bcommand{help}{} to see a list of available commands:

\begin{bconsole}{help}
*<input>help</input>
  Command       Description
  =======       ===========
  add           Add media to a pool
  autodisplay   Autodisplay console messages
  automount     Automount after label
  cancel        Cancel a job
  create        Create DB Pool from resource
  delete        Delete volume, pool or job
  disable       Disable a job
  enable        Enable a job
  estimate      Performs FileSet estimate, listing gives full listing
  exit          Terminate Bconsole session
  export        Export volumes from normal slots to import/export slots
  gui           Non-interactive gui mode
  help          Print help on specific command
  import        Import volumes from import/export slots to normal slots
  label         Label a tape
  list          List objects from catalog
  llist         Full or long list like list command
  messages      Display pending messages
  memory        Print current memory usage
  mount         Mount storage
  move          Move slots in an autochanger
  prune         Prune expired records from catalog
  purge         Purge records from catalog
  quit          Terminate Bconsole session
  query         Query catalog
  restore       Restore files
  relabel       Relabel a tape
  release       Release storage
  reload        Reload conf file
  rerun         Rerun a job
  run           Run a job
  status        Report status
  setbandwidth  Sets bandwidth
  setdebug      Sets debug level
  setip         Sets new client address -- if authorized
  show          Show resource records
  sqlquery      Use SQL to query catalog
  time          Print current time
  trace         Turn on/off trace to file
  unmount       Unmount storage
  umount        Umount - for old-time Unix guys, see unmount
  update        Update volume, pool or stats
  use           Use specific catalog
  var           Does variable expansion
  version       Print Director version
  wait          Wait until no jobs are running
\end{bconsole}

Details of the console program's commands are explained in the
\nameref{sec:bconsole} chapter.

\section{Running a Job}
\label{Running}
\index[general]{Job!Running a}
\index[general]{Running a Job}

At this point, we assume you have done the following:

\begin{itemize}
\item Started the Database
\item Installed Bareos
\item Prepared the database for Bareos
\item Started Bareos Director, Storage Daemon and File Daemon
\item Invoked the Console program with \command{bconsole}
\end{itemize}

Furthermore, we assume for the moment you are using the default configuration
files.

At this point, enter the \bcommand{show}{filesets} and you should get something similar this:

\begin{bconsole}{show filesets}
*<input>show filesets</input>
...
FileSet {
  Name = "SelfTest"
  Include {
    Options {
      Signature = MD5
    }
    File = "/usr/sbin"
  }
}

FileSet {
  Name = "Catalog"
  Include {
    Options {
      Signature = MD5
    }
    File = "/var/lib/bareos/bareos.sql"
    File = "/etc/bareos"
  }
}
...
\end{bconsole}

One of the FileSets is the pre-defined \fileset{SelfTest} FileSet
that will backup the \path|/usr/sbin| directory.
For testing purposes, we have chosen a directory of moderate
size (about 30 Megabytes) and complexity without being too big. The FileSet
\fileset{Catalog} is used for backing up Bareos's catalog and is not of interest
to us for the moment. You can change what is backed up by editing the configuration
and changing the \path|File =| line in the \resourcetype{Dir}{FileSet} resource.

Now is the time to run your first backup job. We are going to backup your
Bareos source directory to a File Volume in your \path|/var/lib/bareos/storage/|
 directory just to show you how easy it is. Now enter:

\begin{bconsole}{status dir}
*<input>status dir</input>
bareos-dir Version: 13.2.0 (09 April 2013) x86_64-pc-linux-gnu debian Debian GNU/Linux 6.0 (squeeze)
Daemon started 23-May-13 13:17. Jobs: run=0, running=0 mode=0
 Heap: heap=270,336 smbytes=59,285 max_bytes=59,285 bufs=239 max_bufs=239

Scheduled Jobs:
Level          Type     Pri  Scheduled          Name               Volume
===================================================================================
Incremental    Backup    10  23-May-13 23:05    BackupClient1      testvol
Full           Backup    11  23-May-13 23:10    BackupCatalog      testvol
====

Running Jobs:
Console connected at 23-May-13 13:34
No Jobs running.
====
\end{bconsole}

where the times and the Director's name will be different according to your
setup. This shows that an Incremental job is scheduled to run for the Job
\resourcename{Dir}{Job}{BackupClient1} at 1:05am and that at 1:10,
a \resourcename{Dir}{Job}{BackupCatalog} is scheduled to
run.
%Note, you should probably change the name \resourcename{Dir}{Job}{BackupClient1} to be the name of
%your machine, if not, when you add additional clients, it will be very
%confusing.

Now enter:

\begin{bconsole}{status client}
*<input>status client</input>
Automatically selected Client: bareos-fd
Connecting to Client bareos-fd at bareos:9102

bareos-fd Version: 13.2.0 (09 April 2013)  x86_64-pc-linux-gnu debian Debian GNU/Linux 6.0 (squeeze)
Daemon started 23-May-13 13:17. Jobs: run=0 running=0.
 Heap: heap=135,168 smbytes=26,000 max_bytes=26,147 bufs=65 max_bufs=66
 Sizeof: boffset_t=8 size_t=8 debug=0 trace=0 bwlimit=0kB/s

Running Jobs:
Director connected at: 23-May-13 13:58
No Jobs running.
====
\end{bconsole}

In this case, the client is named \resourcename{Dir}{Client}{bareos-fd} your name might be different,
but the line beginning with \path|bareos-fd Version| is printed by your \bareosFd,
so we are now sure it is up and running.

Finally do the same for your \bareosSd with:

\begin{bconsole}{status storage}
*<input>status storage</input>
Automatically selected Storage: File
Connecting to Storage daemon File at bareos:9103

bareos-sd Version: 13.2.0 (09 April 2013) x86_64-pc-linux-gnu debian Debian GNU/Linux 6.0 (squeeze)
Daemon started 23-May-13 13:17. Jobs: run=0, running=0.
 Heap: heap=241,664 smbytes=28,574 max_bytes=88,969 bufs=73 max_bufs=74
 Sizes: boffset_t=8 size_t=8 int32_t=4 int64_t=8 mode=0 bwlimit=0kB/s

Running Jobs:
No Jobs running.
====

Device status:

Device "FileStorage" (/var/lib/bareos/storage) is not open.
==
====

Used Volume status:
====

====
\end{bconsole}

You will notice that the default \bareosSd device is named \resourcename{Dir}{Storage}{File} and
that it will use device \directory{/var/lib/bareos/storage}, which is not currently open.

Now, let's actually run a job with:

\footnotesize
\begin{bconsole}{run}
run
\end{bconsole}
\normalsize

you should get the following output:

\footnotesize
\begin{bconsole}{select job}
Automatically selected Catalog: MyCatalog
Using Catalog "MyCatalog"
A job name must be specified.
The defined Job resources are:
     1: BackupClient1
     2: BackupCatalog
     3: RestoreFiles
Select Job resource (1-3):
\end{bconsole}
\normalsize

Here, Bareos has listed the three different Jobs that you can run, and you
should choose number {\bf 1} and type enter, at which point you will get:

\footnotesize
\begin{bconsole}{run job}
Run Backup job
JobName:  BackupClient1
Level:    Incremental
Client:   bareos-fd
Format:   Native
FileSet:  SelfTest
Pool:     Full (From Job resource)
NextPool: *None* (From unknown source)
Storage:  File (From Job resource)
When:     2013-05-23 14:50:04
Priority: 10
OK to run? (yes/mod/no):
\end{bconsole}
\normalsize

At this point, take some time to look carefully at what is printed and
understand it. It is asking you if it is OK to run a job named \job{BackupClient1}
with FileSet \fileset{SelfTest} as an Incremental job on your
Client, and to use Storage \resourcename{Dir}{Storage}{File} and
Pool \pool{Full}, and finally, it wants to run it now (the current time
should be displayed by your console).

Here we have the choice to run ({\bf yes}), to modify one or more of the above
parameters ({\bf mod}), or to not run the job ({\bf no}). Please enter {\bf
yes}, at which point you should immediately get the command prompt (an
asterisk).

If you wait a few seconds, then enter the command \bcommand{messages}{}
you will get back something like:


% \footnotesize
% \begin{bconsole}
% 28-Apr-2003 14:22 bareos-dir: Last FULL backup time not found. Doing
%                   FULL backup.
% 28-Apr-2003 14:22 bareos-dir: Start Backup JobId 1,
%                   Job=Client1.2003-04-28_14.22.33
% 28-Apr-2003 14:22 bareos-sd: Job Client1.2003-04-28_14.22.33 waiting.
%                   Cannot find any appendable volumes.
% Please use the "label"  command to create a new Volume for:
%     Storage:      FileStorage
%     Media type:   File
%     Pool:         Default
% \end{bconsole}
% \normalsize
% 
% The first message, indicates that no previous Full backup was done, so Bareos
% is upgrading our Incremental job to a Full backup (this is normal). The second
% message indicates that the job started with JobId 1., and the third message
% tells us that Bareos cannot find any Volumes in the Pool for writing the
% output. This is normal because we have not yet created (labeled) any Volumes.
% Bareos indicates to you all the details of the volume it needs.
% 
% At this point, the job is BLOCKED waiting for a Volume. You can check this if
% you want by doing a {\bf status dir}. In order to continue, we must create a
% Volume that Bareos can write on. We do so with:
% 
% \footnotesize
% \begin{bconsole}
% label
% \end{bconsole}
% \normalsize
% 
% and Bareos will print:
% 
% \footnotesize
% \begin{bconsole}
% The defined Storage resources are:
%      1: File
% Item 1 selected automatically.
% Enter new Volume name:
% \end{bconsole}
% \normalsize
% 
% at which point, you should enter some name beginning with a letter and
% containing only letters and numbers (period, hyphen, and underscore) are also
% permitted. For example, enter {\bf TestVolume001}, and you should get back:
% 
% \footnotesize
% \begin{bconsole}
% Defined Pools:
%      1: Default
% Item 1 selected automatically.
% Connecting to Storage daemon File at bareos:8103 ...
% Sending label command for Volume "TestVolume001" Slot 0 ...
% 3000 OK label. Volume=TestVolume001 Device=/var/lib/bareos/storage
% Catalog record for Volume "TestVolume002", Slot 0  successfully created.
% Requesting mount FileStorage ...
% 3001 OK mount. Device=/var/lib/bareos/storage
% \end{bconsole}
% \normalsize
%
% Finally, enter {\bf messages} and you should get something like:

\TODO{Replace bconsole output by current version of Bareos.}

\begin{bconsole}{run}
*<input>messages</input>
28-Apr-2003 14:30 bareos-sd: Wrote label to prelabeled Volume
   "TestVolume001" on device /var/lib/bareos/storage
28-Apr-2003 14:30 rufus-dir: Bareos 1.30 (28Apr03): 28-Apr-2003 14:30
JobId:                  1
Job:                    BackupClient1.2003-04-28_14.22.33
FileSet:                Full Set
Backup Level:           Full
Client:                 bareos-fd
Start time:             28-Apr-2003 14:22
End time:               28-Apr-2003 14:30
Files Written:          1,444
Bytes Written:          38,988,877
Rate:                   81.2 KB/s
Software Compression:   None
Volume names(s):        TestVolume001
Volume Session Id:      1
Volume Session Time:    1051531381
Last Volume Bytes:      39,072,359
FD termination status:  OK
SD termination status:  OK
Termination:            Backup OK
28-Apr-2003 14:30 rufus-dir: Begin pruning Jobs.
28-Apr-2003 14:30 rufus-dir: No Jobs found to prune.
28-Apr-2003 14:30 rufus-dir: Begin pruning Files.
28-Apr-2003 14:30 rufus-dir: No Files found to prune.
28-Apr-2003 14:30 rufus-dir: End auto prune.
\end{bconsole}

If you don't see the output immediately, you can keep entering \bcommand{messages}{}
until the job terminates.

Instead of typing \bcommand{messages}{} multiple times,
you can also ask bconsole to wait, until a specific job is finished:

\begin{bconsole}{wait}
*<input>wait jobid=1</input>
\end{bconsole}
or just \bcommand{wait}{}, which waits for all running jobs to finish.

Another useful command is \bcommand{autodisplay}{on}.
With autodisplay activated, messages will automatically be displayed as soon as they are ready.

If you do an \command{ls -l} of your \directory{/var/lib/bareos/storage} directory, you will see that you
have the following item:

\footnotesize
\begin{bconsole}{volume}
-rw-r-----    1 bareos bareos   39072153 Apr 28 14:30 Full-001
\end{bconsole}
\normalsize

This is the file Volume that you just wrote and it contains all the data of
the job just run. If you run additional jobs, they will be appended to this
Volume unless you specify otherwise.

% You might ask yourself if you have to label all the Volumes that Bareos is
% going to use. The answer for disk Volumes, like the one we used, is no. It is
% possible to have Bareos automatically label volumes. For tape Volumes, you
% will most likely have to label each of the Volumes you want to use.

If you would like to stop here, you can simply enter \bcommand{quit}{} in the Console
program.

% To clean up, simply
% delete the file {\bf /tmp/TestVolume001}, and you should also re-initialize
% your database using:
% 
% \footnotesize
% \begin{bconsole}
% ./drop_bareos_tables
% ./make_bareos_tables
% \end{bconsole}
% \normalsize
% 
% Please note that this will erase all information about the previous jobs that
% have run, and that you might want to do it now while testing but that normally
% you will not want to re-initialize your database.

If you would like to try restoring the files that you just backed up, read the
following section.
\label{restoring}

\section{Restoring Your Files}
\index[general]{Files!Restoring Your}
\index[general]{Restoring Your Files}

If you have run the default configuration and run the job as demonstrated above,
you can restore the backed up files in the Console
program by entering:

\begin{bconsole}{restore}
*<input>restore all</input>
First you select one or more JobIds that contain files
to be restored. You will be presented several methods
of specifying the JobIds. Then you will be allowed to
select which files from those JobIds are to be restored.

To select the JobIds, you have the following choices:
     1: List last 20 Jobs run
     2: List Jobs where a given File is saved
     3: Enter list of comma separated JobIds to select
     4: Enter SQL list command
     5: Select the most recent backup for a client
     6: Select backup for a client before a specified time
     7: Enter a list of files to restore
     8: Enter a list of files to restore before a specified time
     9: Find the JobIds of the most recent backup for a client
    10: Find the JobIds for a backup for a client before a specified time
    11: Enter a list of directories to restore for found JobIds
    12: Select full restore to a specified Job date
    13: Cancel
Select item:  (1-13):
\end{bconsole}


As you can see, there are a number of options, but for the current
demonstration, please enter {\bf 5} to do a restore of the last backup you
did, and you will get the following output:

\footnotesize
\begin{bconsole}{select resource}
Automatically selected Client: bareos-fd
The defined FileSet resources are:
     1: Catalog
     2: Full Set
Select FileSet resource (1-2): 
\end{bconsole}
\normalsize

As you can see, Bareos knows what client
you have, and since there was only one, it selected it automatically.
Select {\bf 2}, because you want to restore files from the file set.

\footnotesize
\begin{bconsole}{restore filesystem}
+-------+-------+----------+------------+---------------------+---------------+
| jobid | level | jobfiles | jobbytes   | starttime           | volumename    |
+-------+-------+----------+------------+---------------------+---------------+
|     1 | F     |      166 | 19,069,526 | 2013-05-05 23:05:02 | TestVolume001 |
+-------+-------+----------+------------+---------------------+---------------+
You have selected the following JobIds: 1

Building directory tree for JobId(s) 1 ...  +++++++++++++++++++++++++++++++++++++++++
165 files inserted into the tree and marked for extraction.

You are now entering file selection mode where you add (mark) and
remove (unmark) files to be restored. No files are initially added, unless
you used the "all" keyword on the command line.
Enter "done" to leave this mode.

cwd is: /
$ 
\end{bconsole}
\normalsize

where I have truncated the listing on the right side to make it more readable.

Then Bareos produced a listing containing all the jobs that
form the current backup, in this case, there is only one, and the Storage
daemon was also automatically chosen. Bareos then took all the files that were
in Job number 1 and entered them into a {\bf directory tree} (a sort of in
memory representation of your filesystem). At this point, you can use the 
\bcommand{cd}{} and \bcommand{ls}{} or \bcommand{dir}{} commands to walk up and down the directory tree
and view what files will be restored. For example, if you enter \bcommand{cd}{/usr/sbin}
and then enter \bcommand{dir}{} you will get a listing
of all the files in the \directory{/usr/sbin/} directory.
On your system, the path might be somewhat different.
For more information on this, please refer to the
\ilink{Restore Command Chapter}{RestoreChapter} of this manual for
more details.

To exit this mode, simply enter:

\footnotesize
\begin{bconsole}{done}
done
\end{bconsole}
\normalsize

and you will get the following output:

\footnotesize
\begin{bconsole}{job report}
Bootstrap records written to
   /home/user/bareos/testbin/working/restore.bsr
The restore job will require the following Volumes:

   TestVolume001
1444 files selected to restore.
Run Restore job
JobName:         RestoreFiles
Bootstrap:      /home/user/bareos/testbin/working/restore.bsr
Where:          /tmp/bareos-restores
Replace:        always
FileSet:        Full Set
Backup Client:  rufus-fd
Restore Client: rufus-fd
Storage:        File
JobId:          *None*
When:           2005-04-28 14:53:54
OK to run? (yes/mod/no):
Bootstrap records written to /var/lib/bareos/bareos-dir.restore.1.bsr

The job will require the following
   Volume(s)                 Storage(s)                SD Device(s)
===========================================================================
   
    TestVolume001             File                      FileStorage

Volumes marked with "*" are online.


166 files selected to be restored.

Run Restore job
JobName:         RestoreFiles
Bootstrap:       /var/lib/bareos/bareos-dir.restore.1.bsr
Where:           /tmp/bareos-restores
Replace:         Always
FileSet:         Full Set
Backup Client:   bareos-fd
Restore Client:  bareos-fd
Format:          Native
Storage:         File
When:            2013-05-23 15:56:53
Catalog:         MyCatalog
Priority:        10
Plugin Options:  *None*
OK to run? (yes/mod/no): 
\end{bconsole}
\normalsize

If you answer {\bf yes} your files will be restored to
\directory{/tmp/bareos-restores}. If you want to restore the files to their original
locations, you must use the {\bf mod} option and explicitly set {\bf Where:}
to nothing (or to /). We recommend you go ahead and answer {\bf yes} and after
a brief moment, enter \bcommand{messages}{}, at which point you should get a listing
of all the files that were restored as well as a summary of the job that looks
similar to this:

\footnotesize
\begin{bconsole}{job report}
23-May 15:24 bareos-dir JobId 2: Start Restore Job RestoreFiles.2013-05-23_15.24.01_10
23-May 15:24 bareos-dir JobId 2: Using Device "FileStorage" to read.
23-May 15:24 bareos-sd JobId 2: Ready to read from volume "TestVolume001" on device "FileStorage" (/var/lib/bareos/storage).
23-May 15:24 bareos-sd JobId 2: Forward spacing Volume "TestVolume001" to file:block 0:194.
23-May 15:58 bareos-dir JobId 3: Bareos bareos-dir 13.2.0 (09Apr13):
  Build OS:               x86_64-pc-linux-gnu debian Debian GNU/Linux 6.0 (squeeze)
  JobId:                  2
  Job:                    RestoreFiles.2013-05-23_15.58.48_11
  Restore Client:         bareos-fd
  Start time:             23-May-2013 15:58:50
  End time:               23-May-2013 15:58:52
  Files Expected:         166
  Files Restored:         166
  Bytes Restored:         19,069,526
  Rate:                   9534.8 KB/s
  FD Errors:              0
  FD termination status:  OK
  SD termination status:  OK
  Termination:            Restore OK
\end{bconsole}
\normalsize

After exiting the Console program, you can examine the files in \directory{/tmp/bareos-restores},
which will contain a small directory tree with all the
files. Be sure to clean up at the end with:

\begin{commands}{remove restore directory}
<command>rm</command> -rf /tmp/bareos-restore
\end{commands}

\section{Quitting the Console Program}
\index[general]{Program!Quitting the Console}
\index[general]{Quitting the Console Program}

Simply enter the command \bcommand{quit}{}.

\section{Adding a Client}
\label{SecondClient}
\label{sec:AddAClient}
\index[general]{Client!Adding a Second}
\index[general]{Adding a Client}

If you have gotten the example shown above to work on your system, you may be
ready to add a second Client (\bareosFd). That is you have a second machine
that you would like backed up.
Lets assume, following settings about the machine you want to add to your backup environment:

\begin{description}
  \item[Hostname (FQDN)] \host{client2.example.com}
  \item[IP Address] 192.168.0.2
  \item[OS] Linux (otherwise the paths may differ)
\end{description}

For this you have to make changes on the server side (\bareosDir) and the client side.

\subsection*{Client: install package}

See \nameref{InstallChapter} about how to add the Bareos repository.
The only part you need installed on the other
machine is the \package{bareos-filedaemon}.

\subsection*{Director: configure client}

Bareos \sinceVersion{Dir}{configure add}{16.2.4} offers the \ilink{configure add command}{sec:bcommandConfigure} to add resources to the \bareosDir.

Start the \command{bconsole} and use the \bcommand{configure}{add client} command.
Address must be a DNS resolvable name or an IP address.

\begin{bconsole}{add a client}
*<input>configure add client name=client2-fd address=192.168.0.2 password=secret</input>
Created resource config file "/etc/bareos/bareos-dir.d/client/client2-fd.conf":
Client {
  Name = client2-fd
  Address = 192.168.0.2
  Password = secret
}
\end{bconsole}

This creates two resource configuration files:
\begin{itemize}
  \item \file{/etc/bareos/bareos-dir.d/client/client2-fd.conf}
  \item \file{/etc/bareos/bareos-dir-export/client/client2-fd/bareos-fd.d/director/bareos-dir.conf} (assuming your director resource is named \name{bareos-dir})
\end{itemize}

The \file{/etc/bareos/bareos-dir-export/client/client2-fd/bareos-fd.d/director/bareos-dir.conf} is the required resource needed on the \bareosFd. You can copy it to the destination:

\begin{commands}{Copy the bareos-fd director resource to the new client}
scp /etc/bareos/bareos-dir-export/client/client2-fd/bareos-fd.d/director/bareos-dir.conf root@client2.example.com:/etc/bareos/bareos-fd.d/director/
\end{commands}

\subsubsection*{Manual configuration}

Alternatively you can configure your resources manually.
On the \bareosDir create the file
\begin{bareosConfigResource}{bareos-dir}{client}{client2-fd}
Client {
  Name = client2-fd
  Address = 192.168.0.2
  Password = secret
}
\end{bareosConfigResource}

Reload or restart your \bareosDir:
\begin{bconsole}{reload the Director configuration}
*<input>reload</input>
reloaded
\end{bconsole}

The corresponding \bareosFd director resource can be created directly on the client, see below.


\subsection*{Client: configure}

The package \package{bareos-filedaemon} \sinceVersion{Fd}{Client resource files}{16.2.4} brings several configuration files:
\begin{itemize}
    \item \file{/etc/bareos/bareos-fd.d/client/myself.conf}
    \item \file{/etc/bareos/bareos-fd.d/director/bareos-dir.conf}
    \item \file{/etc/bareos/bareos-fd.d/director/bareos-mon.conf}
    \item \file{/etc/bareos/bareos-fd.d/messages/Standard.conf}
\end{itemize}

In detail:
\begin{description}
    \item[\file{client/myself.conf}] defines the name of the client. The default is \path|<hostname>-fd|.
        Changes are only required, if you want to use another name or en- or disable special \bareosFd features. See \nameref{ClientResourceClient}.

    \item[\file{director/bareos-dir.conf}] gives the \bareosDir \name{bareos-dir} full access to this \bareosFd. During installation, the \linkResourceDirective{Fd}{Director}{Password} is set to a random default. Adapt the name and/or the password to your \bareosDir. (The name \name{bareos-dir} is the default \bareosDir name since Bareos \sinceVersion{Dir}{bareos-dir is the default \bareosDir name}{16.2.4}.)

    \item[\file{director/bareos-mon.conf}] gives the \bareosDir \name{bareos-mon} restricted access to this \bareosFd. During installation, the \linkResourceDirective{Fd}{Director}{Password} is set to a random value. This resource is intended to be used by the local \package{bareos-tray-monitor}.

    \item[\file{messages/Standard.conf}] defines, how messages should be handled. The default sends all relevant messages to the \bareosDir.
\end{description}

If your \bareosDir is named \name{bareos-dir},
the \file{/etc/bareos/bareos-fd.d/director/bareos-dir.conf} may already be overwritten
by the file you copied from the \bareosDir.
If your Director has another name, an addition resource file will exists.
You can define an arbitrary number of \bareosDir's in your \bareosFd configuration.
However, normally you will only have one \resourcetype{Fd}{Director} with full control of your \bareosFd
and optional one \resourcetype{Fd}{Director} for monitoring (used by the \bareosTrayMonitor).

Anyhow, the resource will look similar to this:
\begin{bareosConfigResource}{bareos-fd}{director}{bareos-dir}
Director {
  Name = bareos-dir
  Password = "[md5]5ebe2294ecd0e0f08eab7690d2a6ee69"
}
\end{bareosConfigResource}

After a restart of the \bareosFd to reload the configuration
this resource allows the access for a \bareosDir with name \name{bareos-dir} and password \name{secret} (stored in MD5 format).

% \begin{commands}{restart bareos-fd}
% root@client2:~ # service bareos-fd restart
% \end{commands}
\begin{commands}{restart bareos-fd}
service bareos-fd restart
\end{commands}

\subsubsection*{Manual configuration}

If you have not created the \resourcetype{Fd}{Director} by \bcommand{configure}{}, you can create it also manually.
If your \bareosDir is also named \name{bareos-dir}, modify or create the file \file{/etc/bareos/bareos-fd.d/director/bareos-dir.conf}:

\begin{bareosConfigResource}{bareos-fd}{director}{bareos-dir}
Director {
  Name = "bareos-dir"   # Name of your Bareos Director
  Password = "secret"   # Password (cleartext or MD5) must be identical
                        # to the password of your client reosurce in the Direcotr
                        # (bareos-dir.d/client/client2-fd.conf)
}
\end{bareosConfigResource}

See the relation between resource names and password of the different Bareos components in \nameref{sec:resource-relation}.

If your are not using the \nameref{sec:SubdirectoryConfigurationScheme},
make sure that this resource file gets included in your \bareosFd configuration.
You can verify this by
\begin{commands}{show how bareos-fd would read the current configuration files}
bareos-fd -xc
\end{commands}

After modifying the file, you have to restart the \bareosFd:
\begin{commands}{restart bareos-fd}
service bareos-fd restart
\end{commands}



\subsection*{Director: test client, add a job}

The following example show how to
\begin{itemize}
    \item Verify the network connection from \bareosDir to the \bareosFd.
    \item Add a job resource.
    \item Dry-run the job (\bcommand{estimate}{listing}).
    \item Run the job.
    \item Wait for the job to finish.
    \item Verify the job.
\end{itemize}

\begin{bconsole}{test the client and add a job resource}
*<input>status client=client2-fd</input>
...
*<input>configure add job name=client2-job client=client2-fd jobdefs=DefaultJob</input>
Created resource config file "/etc/bareos/bareos-dir.d/job/client2-job.conf":
Job {
  Name = client2-job
  Client = client2-fd
  JobDefs = DefaultJob
}
*<input>estimate listing job=client2-job</input>
...
*<input>run job=client2-job</input>
...
*<input>wait jobid=...</input>
...
*<input>list joblog jobid=...</input>
...
*<input>list files jobid=...</input>
...
*<input>list volumes</input>
...
\end{bconsole}



% \section{When The Volumes Fills}
% \index[general]{Tape!Full}
% \index[general]{When The Tape Fills}
% \index[general]{Problem!Tape Full}
% 
% If you have scheduled your job, typically nightly, there will come a time when
% the tape fills up and {\bf Bareos} cannot continue. In this case, Bareos will
% send you a message similar to the following:
% 
% \footnotesize
% \begin{bconsole}
% bareos-sd: block.c:337 === Write error errno=28: ERR=No space left on device
% \end{bconsole}
% \normalsize
% 
% This indicates that Bareos got a write error because the tape is full. Bareos
% will then search the Pool specified for your Job looking for an appendable
% volume. In the best of all cases, you will have properly set your Retention
% Periods and you will have all your tapes marked to be Recycled, and {\bf
% Bareos} will automatically recycle the tapes in your pool requesting and
% overwriting old Volumes. For more information on recycling,
% see \nameref{RecyclingChapter}.
% If you
% find that your Volumes were not properly recycled (usually because of a
% configuration error), please see the
% \nameref{manualrecycling} chapter.
% 
% If like me, you have a very large set of Volumes and you label them with the
% date the Volume was first writing, or you have not set up your Retention
% periods, Bareos will not find a tape in the pool, and it will send you a
% message similar to the following:
% 
% \footnotesize
% \begin{bconsole}
% bareos-sd: Job usersave.2002-09-19.10:50:48 waiting. Cannot find any
%           appendable volumes.
% Please use the "label"  command to create a new Volume for:
%     Storage:      SDT-10000
%     Media type:   DDS-4
%     Pool:         Default
% \end{bconsole}
% \normalsize
% 
% Until you create a new Volume, this message will be repeated an hour later,
% then two hours later, and so on doubling the interval each time up to a
% maximum interval of one day.
% 
% The obvious question at this point is: What do I do now?
% 
% The answer is simple: first, using the Console program, close the tape drive
% using the {\bf unmount} command. If you only have a single drive, it will be
% automatically selected, otherwise, make sure you release the one specified on
% the message (in this case {\bf STD-10000}).
% 
% Next, you remove the tape from the drive and insert a new blank tape.
% % Note, on some older tape drives, you may need to write an end of file mark
% % (\command{mt -f /dev/nst0 weof}) to prevent the drive
% % from running away when Bareos attempts to read the label.
% 
% Finally, you use the {\bf label} command in the Console to write a label to
% the new Volume. The {\bf label} command will contact the Storage daemon to
% write the software label, if it is successful, it will add the new Volume to
% the Pool, then issue a {\bf mount} command to the Storage daemon. See the
% previous sections of this chapter for more details on labeling tapes.
% 
% The result is that Bareos will continue the previous Job writing the backup to
% the new Volume.
% 
% If you have a Pool of volumes and Bareos is cycling through them, instead of
% the above message "Cannot find any appendable volumes.", Bareos may ask you
% to mount a specific volume. In that case, you should attempt to do just that.
% If you do not have the volume any more (for any of a number of reasons), you
% can simply mount another volume from the same Pool, providing it is
% appendable, and Bareos will use it. You can use the {\bf list volumes} command
% in the console program to determine which volumes are appendable and which are
% not.
% 
% If like me, you have your Volume retention periods set correctly, but you have
% no more free Volumes, you can relabel and reuse a Volume as follows:
% 
% \begin{itemize}
% \item Do a {\bf list volumes} in the Console and select the oldest  Volume for
%    relabeling.
% \item If you have setup your Retention periods correctly, the  Volume should
%    have VolStatus {\bf Purged}.
% \item If the VolStatus is not set to Purged, you will need to purge  the
%    database of Jobs that are written on that Volume. Do so  by using the command
%    {\bf purge jobs volume} in the Console.  If you have multiple Pools, you will
% be prompted for the  Pool then enter the VolumeName (or MediaId) when
% requested.
% \item Then simply use the {\bf relabel} command to relabel the  Volume.
%    \end{itemize}
% 
% To manually relabel the Volume use the following additional steps:
% 
% \begin{itemize}
% \item To delete the Volume from the catalog use the {\bf delete volume}
%    command in the Console and select the VolumeName (or MediaId) to be  deleted.
% 
% \item Use the {\bf unmount} command in the Console to unmount the  old tape.
% \item Physically relabel the old Volume that you deleted so that it  can be
%    reused.
% \item Insert the old Volume in the tape drive.
% \item From a command line do: {\bf mt \ -f \ /dev/st0 \ rewind} and  {\bf mt \
%    -f \ /dev/st0 \ weof}, where you need to use the proper  tape drive name for
%    your system in place of {\bf /dev/st0}.
% \item Use the {\bf label} command in the Console to write a new  Bareos label
%    on your tape.
% \item Use the {\bf mount} command in the Console if it is not automatically
%    done, so that Bareos starts using your newly labeled tape.
%    \end{itemize}


% \section{Debug Daemon Output}
% \index[general]{Debug!Daemon}
% \index[general]{Daemon!Debug}
%
% %\TODO{needs to be adapted}
%
% If you want debug output from the daemons as they are running, start the
% daemons from the install directory as follows:
% 
% \footnotesize
% \begin{bconsole}
% ./bareos start -d100
% \end{bconsole}
% \normalsize
% 
% This can be particularly helpful if your daemons do not start correctly,
% because direct daemon output to the console is normally directed to the
% NULL device, but with the debug level greater than zero, the output
% will be sent to the starting terminal.
% 
% To stop the three daemons, enter the following from the install directory:
% 
% \footnotesize
% \begin{bconsole}
% ./bareos stop
% \end{bconsole}
% \normalsize
% 
% The execution of {\bf bareos stop} may complain about pids not found. This is
% OK, especially if one of the daemons has died, which is very rare.
% 
% To do a full system save, each File daemon must be running as root so that it
% will have permission to access all the files. None of the other daemons
% require root privileges. However, the Storage daemon must be able to open the
% tape drives. On many systems, only root can access the tape drives. Either run
% the Storage daemon as root, or change the permissions on the tape devices to
% permit non-root access. MySQL and PostgreSQL can be installed and run with any
% userid; root privilege is not necessary.

\section{Patience When Starting Daemons or Mounting Blank Tapes}

When you start the Bareos daemons, the Storage daemon attempts to open all
defined storage devices and verify the currently mounted Volume (if
configured). Until all the storage devices are verified, the Storage daemon
will not accept connections from the Console program. If a tape was previously
used, it will be rewound, and on some devices this can take several minutes.
As a consequence, you may need to have a bit of patience when first contacting
the Storage daemon after starting the daemons. If you can see your tape drive,
once the lights stop flashing, the drive will be ready to be used.

The same considerations apply if you have just mounted a blank tape in a drive.
It can take a minute or two before the drive properly
recognizes that the tape is blank.
If you attempt to \bcommand{mount}{} the tape with
the Console program during this recognition period, it is quite possible that
you will hang your SCSI driver.
As a
consequence, you are again urged to have patience when inserting blank tapes.
Let the device settle down before attempting to access it.



\section{Pools}
\label{Pool}
\index[general]{Pool!Overview}

Creating the Pool is automatically done when the \bareosDir starts, so if you
understand Pools, you can skip to the next section.

When you run a backup job, one of the things that Bareos must know is what Volumes to
use. Instead of specifying a Volume (tape) directly, you
specify which Pool of Volumes you want Bareos to consult when it wants a Volume
for writing backups. Bareos will select the first available Volume from the
Pool that is appropriate for the \linkResourceDirective{Dir}{Job}{Storage} you have specified for the Job
being run. When a volume has filled up with data, Bareos will change its
\name{VolStatus} from \name{Append} to \name{Full}, and then Bareos will use the
next volume and so on. If no appendable Volume exists in the Pool, the
Director will attempt to recycle an old Volume.
For details, please read the \nameref{RecyclingChapter} chapter.

If there are still no
appendable Volumes available, Bareos will send a message requesting the
operator to create an appropriate Volume.

Bareos keeps track of the Pool name, the volumes contained in the Pool,
and a number of attributes of each of those Volumes.

When Bareos starts, it ensures that all Pool resource definitions have been
recorded in the catalog. You can verify this by entering:
\begin{bconsole}{list pools}
*<input>list pools</input>
+--------+--------------+---------+---------+----------+---------------+
| PoolId | Name         | NumVols | MaxVols | PoolType | LabelFormat   |
+--------+--------------+---------+---------+----------+---------------+
| 1      | Full         | 1       | 100     | Backup   | Full-         |
| 2      | Differential | 0       | 100     | Backup   | Differential- |
| 3      | Incremental  | 1       | 100     | Backup   | Incremental-  |
| 4      | Scratch      | 0       | 0       | Backup   | *             |
+--------+--------------+---------+---------+----------+---------------+
\end{bconsole}


% \section{Labeling Your Volumes}
% \index[general]{Volume!Label}
% \index[general]{Label!Labeling Volumes}
% 
% Bareos requires that each Volume contains a software label. There are several
% strategies for labeling volumes. The one I use is to label them as they are
% needed by {\bf Bareos} using the console program. That is when Bareos needs a
% new Volume, and it does not find one in the catalog, it will send me an email
% message requesting that I add Volumes to the Pool. I then use the {\bf label}
% command in the Console program to label a new Volume and to define it in the
% Pool database, after which Bareos will begin writing on the new Volume.
% Alternatively, I can use the Console {\bf relabel} command to relabel a Volume
% that is no longer used providing it has VolStatus {\bf Purged}.
% 
% Another strategy is to label a set of volumes at the start, then use them as
% {\bf Bareos} requests them. This is most often done if you are cycling through
% a set of tapes, for example using an autochanger. For more details on
% recycling, please see the
% \ilink{Automatic Volume Recycling}{RecyclingChapter} chapter of
% this manual.
% 
% If you run a Bareos job, and you have no labeled tapes in the Pool, Bareos
% will inform you, and you can create them "on-the-fly" so to speak. In my
% case, I label my tapes with the date, for example: {\bf DLT-18April02}. See
% below for the details of using the {\bf label} command.
% 
% \subsection*{Labeling Volumes with the Console Program}
% \index[general]{Label!Labeling Volumes with the Console Program}
% \index[general]{Console!Command!label}
% 
% Labeling volumes is normally done by using the console program.
% 
% \begin{enumerate}
% \item bconsole
% \item label
% \end{enumerate}
% 
% If Bareos complains that you cannot label the tape because it is already
% labeled, simply {\bf unmount} the tape using the {\bf unmount} command in the
% console, then physically mount a blank tape and re-issue the {\bf label}
% command.
% 
% Since the physical storage media is different for each device, the {\bf label}
% command will provide you with a list of the defined Storage resources such as
% the following:
% 
% \footnotesize
% \begin{bconsole}
% The defined Storage resources are:
%      1: File
%      2: 8mmDrive
%      3: DLTDrive
%      4: SDT-10000
% Select Storage resource (1-4):
% \end{bconsole}
% \normalsize
% 
% At this point, you should have a blank tape in the drive corresponding to the
% Storage resource that you select.
% 
% It will then ask you for the Volume name.
% 
% \footnotesize
% \begin{bconsole}
% Enter new Volume name:
% \end{bconsole}
% \normalsize
% 
% If Bareos complains:
% 
% \footnotesize
% \begin{bconsole}
% Media record for Volume xxxx already exists.
% \end{bconsole}
% \normalsize
% 
% It means that the volume name {\bf xxxx} that you entered already exists in
% the Media database. You can list all the defined Media (Volumes) with the {\bf
% list media} command. Note, the LastWritten column has been truncated for
% proper printing.
% 
% \footnotesize
% \begin{bconsole}
% +---------------+---------+--------+----------------+-----/~/-+------------+-----+
% | VolumeName    | MediaTyp| VolStat| VolBytes       | LastWri | VolReten   | Recy|
% +---------------+---------+--------+----------------+---------+------------+-----+
% | DLTVol0002    | DLT8000 | Purged | 56,128,042,217 | 2001-10 | 31,536,000 |   0 |
% | DLT-07Oct2001 | DLT8000 | Full   | 56,172,030,586 | 2001-11 | 31,536,000 |   0 |
% | DLT-08Nov2001 | DLT8000 | Full   | 55,691,684,216 | 2001-12 | 31,536,000 |   0 |
% | DLT-01Dec2001 | DLT8000 | Full   | 55,162,215,866 | 2001-12 | 31,536,000 |   0 |
% | DLT-28Dec2001 | DLT8000 | Full   | 57,888,007,042 | 2002-01 | 31,536,000 |   0 |
% | DLT-20Jan2002 | DLT8000 | Full   | 57,003,507,308 | 2002-02 | 31,536,000 |   0 |
% | DLT-16Feb2002 | DLT8000 | Full   | 55,772,630,824 | 2002-03 | 31,536,000 |   0 |
% | DLT-12Mar2002 | DLT8000 | Full   | 50,666,320,453 | 1970-01 | 31,536,000 |   0 |
% | DLT-27Mar2002 | DLT8000 | Full   | 57,592,952,309 | 2002-04 | 31,536,000 |   0 |
% | DLT-15Apr2002 | DLT8000 | Full   | 57,190,864,185 | 2002-05 | 31,536,000 |   0 |
% | DLT-04May2002 | DLT8000 | Full   | 60,486,677,724 | 2002-05 | 31,536,000 |   0 |
% | DLT-26May02   | DLT8000 | Append |  1,336,699,620 | 2002-05 | 31,536,000 |   1 |
% +---------------+---------+--------+----------------+-----/~/-+------------+-----+
% \end{bconsole}
% \normalsize
% 
% Once Bareos has verified that the volume does not already exist, it will
% prompt you for the name of the Pool in which the Volume (tape) is to be
% created.  If there is only one Pool (Default), it will be automatically
% selected.
% 
% If the tape is successfully labeled, a Volume record will also be created in
% the Pool. That is the Volume name and all its other attributes will appear
% when you list the Pool. In addition, that Volume will be available for backup
% if the MediaType matches what is requested by the Storage daemon.
% 
% When you labeled the tape, you answered very few questions about it --
% principally the Volume name, and perhaps the Slot. However, a Volume record in
% the catalog database (internally known as a Media record) contains quite a few
% attributes. Most of these attributes will be filled in from the default values
% that were defined in the Pool (i.e. the Pool holds most of the default
% attributes used when creating a Volume).
% 
% It is also possible to add media to the pool without physically labeling the
% Volumes. This can be done with the \bcommand{add}{} command. For more information,
% please see the
% \nameref{sec:bconsole} chapter.

\section{Other Useful Console Commands}
\index[general]{Console!Commands!Useful}

\begin{description}
\item [help]
    Show the list all all available commands.

\item [help list]
    Show detail information about a specific command, in this case the command \bcommand{list}{}.

\item [status dir]
   \index[general]{Console!Command!status dir}
   Print a status of all running jobs and jobs  scheduled in the next 24 hours.

\item [status]
   \index[general]{Console!Command!status}
   The console program will prompt you to select  a daemon type, then will
request the daemon's status.

\item [status jobid=nn]
   \index[general]{Console!Command!status jobid}
   Print a status of JobId nn if it is running.  The Storage daemon is contacted
and requested to print a current  status of the job as well.

\item [list pools]
   \index[general]{Console!Command!list pools}
   List the pools defined in the Catalog (normally  only Default is used).

\item [list volumes]
   \index[general]{Console!Command!list volumes}
   Lists all the media defined in the Catalog.

\item [list jobs]
   \index[general]{Console!Command!list jobs}
   Lists all jobs in the Catalog that have run.

\item [list jobid=nn]
   \index[general]{Console!Command!list jobid}
   Lists JobId nn from the Catalog.

\item [list jobtotals]
   \index[general]{Console!Command!list jobtotals}
   Lists totals for all jobs in the Catalog.

\item [list files jobid=nn]
   \index[general]{Console!Command!list files jobid}
   List the files that were saved for JobId nn.

\item [list jobmedia]
   \index[general]{Console!Command!list jobmedia}
   List the media information for each Job run.

\item [messages]
   \index[general]{Console!Command!messages}
   Prints any messages that have been directed to the console.

% \item [unmount storage=storage-name]
%    \index[general]{Console!Command!unmount storage}
%    Unmounts the drive associated with the storage  device with the name {\bf
% storage-name} if the drive is not currently being  used. This command is used
% if you wish Bareos to free the drive so  that you can use it to label a tape.

% \item [mount storage=storage-name]
%    \index[general]{Console!Command!mount storage}
%    Causes the drive associated with the  storage device to be mounted again. When
% Bareos reaches the end of a volume and requests you to mount a  new volume,
% you must issue this command after you have placed the  new volume in the
% drive. In effect, it is the signal needed by  Bareos to know to start reading
% or writing the new volume.

\item [quit]
   \index[general]{Console!Command!quit}
   Exit or quit the console program.
\end{description}

Most of the commands given above, with the exception of {\bf list}, will
prompt you for the necessary arguments if you simply enter the command name.

The full list of commands is shown in the chapter \nameref{sec:ConsoleCommands}.
