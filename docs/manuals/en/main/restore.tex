%%
%%
\chapter{The Restore Command}
\label{RestoreChapter}
\index[general]{Restore}

\section{General}

Below, we will discuss restoring files with the Console {\bf restore} command,
which is the recommended way of doing restoring files. It is not possible
to restore files by automatically starting a job as you do with Backup,
Verify, ... jobs.  However, in addition to the console restore command,
there is a standalone program named {\bf bextract}, which also permits
restoring files.  For more information on this program, please see the
\ilink{Bareos Utility Programs}{bextract} chapter of this manual. We
don't particularly recommend the {\bf bextract} program because it
lacks many of the features of the normal Bareos restore, such as the
ability to restore Win32 files to Unix systems, and the ability to
restore access control lists (ACL).  As a consequence, we recommend,
wherever possible to use Bareos itself for restores as described below.

You may also want to look at the {\bf bls} program in the same chapter,
which allows you to list the contents of your Volumes.  Finally, if you
have an old Volume that is no longer in the catalog, you can restore the
catalog entries using the program named {\bf bscan}, documented in the same
\ilink{Bareos Utility Programs}{bscan} chapter.

In general, to restore a file or a set of files, you must run a {\bf restore}
job. That is a job with {\bf Type = Restore}. As a consequence, you will need
a predefined {\bf restore} job in your {\bf bareos-dir.conf} (Director's
config) file. The exact parameters (Client, FileSet, ...) that you define are
not important as you can either modify them manually before running the job or
if you use the {\bf restore} command, explained below, Bareos will
automatically set them for you. In fact, you can no longer simply run a restore
job.  You must use the restore command.

Since Bareos is a network backup program, you must be aware that when you
restore files, it is up to you to ensure that you or Bareos have selected the
correct Client and the correct hard disk location for restoring those files.
{\bf Bareos} will quite willingly backup client A, and restore it by sending
the files to a different directory on client B. Normally, you will want to
avoid this, but assuming the operating systems are not too different in their
file structures, this should work perfectly well, if so desired.
By default, Bareos will restore data to the same Client that was backed
up, and those data will be restored not to the original places but to
{\bf /tmp/bareos-restores}. This is configured in the default restore
command resource in bareos-dir.conf.  You may modify any of these defaults when the
restore command prompts you to run the job by selecting the {\bf mod}
option.

\section{The Restore Command}
\index[general]{Console!Command!restore}

Since Bareos maintains a catalog of your files and on which Volumes (disk or
tape), they are stored, it can do most of the bookkeeping work, allowing you
simply to specify what kind of restore you want (current, before a particular
date), and what files to restore. Bareos will then do the rest.

This is accomplished using the {\bf restore} command in the Console. First you
select the kind of restore you want, then the JobIds are selected,
the File records for those Jobs are placed in an internal Bareos directory
tree, and the restore enters a file selection mode that allows you to
interactively walk up and down the file tree selecting individual files to be
restored. This mode is somewhat similar to the standard Unix {\bf restore}
program's interactive file selection mode.

If a Job's file records have been pruned from the catalog, the {\bf restore}
command will be unable to find any files to restore. Bareos will ask if you
want to restore all of them or if you want to use a regular expression to
restore only a selection while reading media.
See \ilink{FileRegex option}{FileRegex} and below for more details on this.

Within the Console program, after entering the {\bf restore} command, you are
presented with the following selection prompt:

\begin{bconsole}{restore}
* <input>restore</input>
First you select one or more JobIds that contain files
to be restored. You will be presented several methods
of specifying the JobIds. Then you will be allowed to
select which files from those JobIds are to be restored.

To select the JobIds, you have the following choices:
     1: List last 20 Jobs run
     2: List Jobs where a given File is saved
     3: Enter list of comma separated JobIds to select
     4: Enter SQL list command
     5: Select the most recent backup for a client
     6: Select backup for a client before a specified time
     7: Enter a list of files to restore
     8: Enter a list of files to restore before a specified time
     9: Find the JobIds of the most recent backup for a client
    10: Find the JobIds for a backup for a client before a specified time
    11: Enter a list of directories to restore for found JobIds
    12: Select full restore to a specified Job date
    13: Cancel
Select item:  (1-13):
\end{bconsole}

There are a lot of options, and as a point of reference, most people will
want to select item 5 (the most recent backup for a client). The details
of the above options are:

\begin{itemize}
\item Item 1 will list the last 20 jobs run. If you find the Job you want,
   you can then select item 3 and enter its JobId(s).

\item Item 2 will list all the Jobs where a specified file is saved.  If you
   find the Job you want, you can then select item 3 and enter the JobId.

\item Item 3 allows you the enter a list of comma separated JobIds whose
   files will be put into the directory tree. You may then select which
   files from those JobIds to restore. Normally, you would use this option
   if you have a particular version of a file that you want to restore and
   you know its JobId. The most common options (5 and 6) will not select
   a job that did not terminate normally, so if you know a file is
   backed up by a Job that failed (possibly because of a system crash), you
   can access it through this option by specifying the JobId.

\item Item 4 allows you to enter any arbitrary SQL command.  This is
   probably the most primitive way of finding the desired JobIds, but at
   the same time, the most flexible.  Once you have found the JobId(s), you
   can select item 3 and enter them.

\item Item 5 will automatically select the most recent Full backup and all
   subsequent incremental and differential backups for a specified Client.
   These are the Jobs and Files which, if reloaded, will restore your
   system to the most current saved state.  It automatically enters the
   JobIds found into the directory tree in an optimal way such that only
   the most recent copy of any particular file found in the set of Jobs
   will be restored.  This is probably the most convenient of all the above
   options to use if you wish to restore a selected Client to its most
   recent state.

   There are two important things to note. First, this automatic selection
   will never select a job that failed (terminated with an error status).
   If you have such a job and want to recover one or more files from it,
   you will need to explicitly enter the JobId in item 3, then choose the
   files to restore.

   If some of the Jobs that are needed to do the restore have had their
   File records pruned, the restore will be incomplete. Bareos currently
   does not correctly detect this condition.  You can however, check for
   this by looking carefully at the list of Jobs that Bareos selects and
   prints. If you find Jobs with the JobFiles column set to zero, when
   files should have been backed up, then you should expect problems.

   If all the File records have been pruned, Bareos will realize that there
   are no file records in any of the JobIds chosen and will inform you. It
   will then propose doing a full restore (non-selective) of those JobIds.
   This is possible because Bareos still knows where the beginning of the
   Job data is on the Volumes, even if it does not know where particular
   files are located or what their names are.

\item Item 6 allows you to specify a date and time, after which Bareos  will
   automatically select the most recent Full backup and all  subsequent
   incremental and differential backups that started  before the specified date
   and time.

\item Item 7 allows you to specify one or more filenames  (complete path
   required) to be restored. Each filename  is entered one at a time or if you
   prefix a filename  with the less-than symbol ({\textless}) Bareos will read that
   file and assume it is a list of filenames to be restored.  If you
   prefix the filename with a question mark (?), then the filename will
   be interpreted as an SQL table name, and Bareos will include the rows
   of that table in the list to be restored. The table must contain the
   JobId in the first column and the FileIndex in the second column.
   This table feature is intended for external programs that want to build
   their own list of files to be restored.
   The filename entry mode is terminated by entering a  blank line.

\item Item 8 allows you to specify a date and time before  entering the
   filenames. See Item 7 above for more  details.

\item Item 9 allows you find the JobIds of the most recent backup for
   a client. This is much like option 5 (it uses the same code), but
   those JobIds are retained internally as if you had entered them
   manually.  You may then select item 11 (see below) to restore one
   or more directories.

\item Item 10 is the same as item 9, except that it allows you to enter
   a before date (as with item 6). These JobIds will then be retained
   internally.

\index[general]{Restore Directories}
\item Item 11 allows you to enter a list of JobIds from which you can
   select directories to be restored. The list of JobIds can have been
   previously created by using either item 9 or 10 on the menu.  You
   may then enter a full path to a directory name or a filename preceded
   by a less than sign ({\textless}). The filename should contain a list
   of directories to be restored.  All files in those directories will
   be restored, but if the directory contains subdirectories, nothing
   will be restored in the subdirectory unless you explicitly enter its
   name.

\item Item 12 is a full restore to a specified job date.

\item Item 13 allows you to cancel the restore command.
\end{itemize}

As an example, suppose that we select item 5 (restore to most recent state).
If you have not specified a client=xxx on the command line, it
it will then ask for the desired Client, which on my system, will print all
the Clients found in the database as follows:

\begin{bconsole}{restore: select client}
Select item:  (1-13): <input>5</input>
Defined clients:
     1: Rufus
     2: Matou
     3: Polymatou
     4: Minimatou
     5: Minou
     6: MatouVerify
     7: PmatouVerify
     8: RufusVerify
     9: Watchdog
Select Client (File daemon) resource (1-9): <input>1</input>
\end{bconsole}

The listed clients are only examples, yours will look differently.
If you have only one Client, it will be automatically selected.
In this example, I enter 1 for
{\bf Rufus} to select the Client.
Then Bareos needs to know what FileSet is
to be restored, so it prompts with:

\footnotesize
\begin{verbatim}
The defined FileSet resources are:
     1: Full Set
     2: Other Files
Select FileSet resource (1-2):
\end{verbatim}
\normalsize

If you have only one FileSet defined for the Client, it will be selected
automatically.  I choose item 1, which is my full backup.  Normally, you
will only have a single FileSet for each Job, and if your machines are
similar (all Linux) you may only have one FileSet for all your Clients.

At this point, Bareos has all the information it needs to find the most
recent set of backups. It will then query the database, which may take a bit
of time, and it will come up with something like the following. Note, some of
the columns are truncated here for presentation:

\footnotesize
\begin{verbatim}
+-------+------+----------+-------------+-------------+------+-------+------------+
| JobId | Levl | JobFiles | StartTime   | VolumeName  | File | SesId |VolSesTime  |
+-------+------+----------+-------------+-------------+------+-------+------------+
| 1,792 | F    |  128,374 | 08-03 01:58 | DLT-19Jul02 |   67 |    18 | 1028042998 |
| 1,792 | F    |  128,374 | 08-03 01:58 | DLT-04Aug02 |    0 |    18 | 1028042998 |
| 1,797 | I    |      254 | 08-04 13:53 | DLT-04Aug02 |    5 |    23 | 1028042998 |
| 1,798 | I    |       15 | 08-05 01:05 | DLT-04Aug02 |    6 |    24 | 1028042998 |
+-------+------+----------+-------------+-------------+------+-------+------------+
You have selected the following JobId: 1792,1792,1797
Building directory tree for JobId 1792 ...
Building directory tree for JobId 1797 ...
Building directory tree for JobId 1798 ...
cwd is: /
$
\end{verbatim}
\normalsize

Depending on the number of {\bf JobFiles} for each JobId,
the \bquote{\bconsoleOutput{Building directory tree ...}} can take a bit of time.
If you notice ath all the
JobFiles are zero, your Files have probably been pruned and you will not be
able to select any individual files -- it will be restore everything or
nothing.

In our example, Bareos found four Jobs that comprise the most recent backup of
the specified Client and FileSet. Two of the Jobs have the same JobId because
that Job wrote on two different Volumes. The third Job was an incremental
backup to the previous Full backup, and it only saved 254 Files compared to
128,374 for the Full backup. The fourth Job was also an incremental backup
that saved 15 files.

Next Bareos entered those Jobs into the directory tree, with no files marked
to be restored as a default, tells you how many files are in the tree, and
tells you that the current working directory ({\bf cwd}) is /. Finally, Bareos
prompts with the dollar sign (\$) to indicate that you may enter commands to
move around the directory tree and to select files.

If you want all the files to automatically be marked when the directory
tree is built, you could have entered the command {\bf restore all}, or
at the \$ prompt, you can simply enter {\bf mark *}.

Instead of choosing item 5 on the first menu (Select the most recent backup
for a client), if we had chosen item 3 (Enter list of JobIds to select) and we
had entered the JobIds {\bf 1792,1797,1798} we would have arrived at the same
point.

One point to note, if you are manually entering JobIds, is that you must enter
them in the order they were run (generally in increasing JobId order). If you
enter them out of order and the same file was saved in two or more of the
Jobs, you may end up with an old version of that file (i.e. not the most
recent).

Directly entering the JobIds can also permit you to recover data from
a Job that wrote files to tape but that terminated with an error status.

While in file selection mode, you can enter {\bf help} or a question mark (?)
to produce a summary of the available commands:

\footnotesize
\begin{verbatim}
 Command    Description
  =======    ===========
  cd         change current directory
  count      count marked files in and below the cd
  dir        long list current directory, wildcards allowed
  done       leave file selection mode
  estimate   estimate restore size
  exit       same as done command
  find       find files, wildcards allowed
  help       print help
  ls         list current directory, wildcards allowed
  lsmark     list the marked files in and below the cd
  mark       mark dir/file to be restored recursively in dirs
  markdir    mark directory name to be restored (no files)
  pwd        print current working directory
  unmark     unmark dir/file to be restored recursively in dir
  unmarkdir  unmark directory name only no recursion
  quit       quit and do not do restore
  ?          print help
\end{verbatim}
\normalsize

As a default no files have been selected for restore (unless you
added {\bf all} to the command line. If you want to restore
everything, at this point, you should enter {\bf mark *}, and then {\bf done}
and Bareos will write the bootstrap records to a file and request your
approval to start a restore job.

If you do not enter the above mentioned {\bf mark *} command, you will start
with an empty state. Now you can simply start looking at the tree and {\bf
mark} particular files or directories you want restored. It is easy to make
a mistake in specifying a file to mark or unmark, and Bareos's error handling
is not perfect, so please check your work by using the {\bf ls} or {\bf dir}
commands to see what files are actually selected. Any selected file has its
name preceded by an asterisk.

To check what is marked or not marked, enter the {\bf count} command, which
displays:

\footnotesize
\begin{verbatim}
128401 total files. 128401 marked to be restored.

\end{verbatim}
\normalsize

Each of the above commands will be described in more detail in the next
section. We continue with the above example, having accepted to restore all
files as Bareos set by default. On entering the {\bf done} command, Bareos
prints:

\footnotesize
\begin{verbatim}
Run Restore job
JobName:         RestoreFiles
Bootstrap:       /var/lib/bareos/client1.restore.3.bsr
Where:           /tmp/bareos-restores
Replace:         Always
FileSet:         Full Set
Backup Client:   client1
Restore Client:  client1
Format:          Native
Storage:         File
When:            2013-06-28 13:30:08
Catalog:         MyCatalog
Priority:        10
Plugin Options:  *None*
OK to run? (yes/mod/no):
\end{verbatim}
\normalsize

Please examine each of the items very carefully to make sure that they are
correct.  In particular, look at {\bf Where}, which tells you where in the
directory structure the files will be restored, and {\bf Client}, which
tells you which client will receive the files.  Note that by default the
Client which will receive the files is the Client that was backed up.
These items will not always be completed with the correct values depending
on which of the restore options you chose.  You can change any of these
default items by entering {\bf mod} and responding to the prompts.

The above assumes that you have defined a {\bf Restore} Job resource in your
Director's configuration file. Normally, you will only need one Restore Job
resource definition because by its nature, restoring is a manual operation,
and using the Console interface, you will be able to modify the Restore Job to
do what you want.

An example Restore Job resource definition is given below.

Returning to the above example, you should verify that the Client name is
correct before running the Job. However, you may want to modify some of the
parameters of the restore job. For example, in addition to checking the Client
it is wise to check that the Storage device chosen by Bareos is indeed
correct. Although the {\bf FileSet} is shown, it will be ignored in restore.
The restore will choose the files to be restored either by reading the {\bf
Bootstrap} file, or if not specified, it will restore all files associated
with the specified backup {\bf JobId} (i.e. the JobId of the Job that
originally backed up the files).

Finally before running the job, please note that the default location for
restoring files is {\bf not} their original locations, but rather the directory
{\bf /tmp/bareos-restores}. You can change this default by modifying your {\bf
bareos-dir.conf} file, or you can modify it using the {\bf mod} option. If you
want to restore the files to their original location, you must have {\bf
Where} set to nothing or to the root, i.e. {\bf /}.

If you now enter {\bf yes}, Bareos will run the restore Job.


\section{Selecting Files by Filename}
\index[general]{Restore!by filename}


If you have a small number of files to restore, and you know the filenames,
you can either put the list of filenames in a file to be read by Bareos, or
you can enter the names one at a time. The filenames must include the full
path and filename. No wild cards are used.

To enter the files, after the {\bf restore}, you select item number 7 from the
prompt list:

\begin{bconsole}{restore list of files}
* <input>restore</input>
First you select one or more JobIds that contain files
to be restored. You will be presented several methods
of specifying the JobIds. Then you will be allowed to
select which files from those JobIds are to be restored.

To select the JobIds, you have the following choices:
     1: List last 20 Jobs run
     2: List Jobs where a given File is saved
     3: Enter list of comma separated JobIds to select
     4: Enter SQL list command
     5: Select the most recent backup for a client
     6: Select backup for a client before a specified time
     7: Enter a list of files to restore
     8: Enter a list of files to restore before a specified time
     9: Find the JobIds of the most recent backup for a client
    10: Find the JobIds for a backup for a client before a specified time
    11: Enter a list of directories to restore for found JobIds
    12: Select full restore to a specified Job date
    13: Cancel
Select item:  (1-13): <input>7</input>
\end{bconsole}


which then prompts you for the client name:

\footnotesize
\begin{verbatim}
Defined Clients:
     1: client1
     2: Tibs
     3: Rufus
Select the Client (1-3): 3
\end{verbatim}
\normalsize

Of course, your client list will be different, and if you have only one
client, it will be automatically selected. And finally, Bareos requests you to
enter a filename:

\footnotesize
\begin{verbatim}
Enter filename:
\end{verbatim}
\normalsize

At this point, you can enter the full path and filename

\footnotesize
\begin{verbatim}
Enter filename: /etc/resolv.conf
Enter filename:
\end{verbatim}
\normalsize

as you can see, it took the filename. If Bareos cannot find a copy of the
file, it prints the following:

\footnotesize
\begin{verbatim}
Enter filename: junk filename
No database record found for: junk filename
Enter filename:
\end{verbatim}
\normalsize

If you want Bareos to read the filenames from a file, you simply precede the
filename with a less-than symbol ({\textless}).

It is possible to automate the selection by file by putting your list of files
in say {\bf /tmp/file-list}, then using the following command:

\footnotesize
\begin{verbatim}
restore client=client1 file=</tmp/file-list
\end{verbatim}
\normalsize

If in modifying the parameters for the Run Restore job, you find that Bareos
asks you to enter a Job number, this is because you have not yet specified
either a Job number or a Bootstrap file. Simply entering zero will allow you
to continue and to select another option to be modified.

\label{Replace}

\section{Replace Options}

When restoring, you have the option to specify a Replace option.  This
directive determines the action to be taken when restoring a file or
directory that already exists.  This directive can be set by selecting
the {\bf mod} option.  You will be given a list of parameters to choose
from.  Full details on this option can be found in the Job Resource section
of the Director documentation.


\section{Command Line Arguments}
\label{CommandArguments}

If all the above sounds complicated, you will probably agree that it really
isn't after trying it a few times. It is possible to do everything that was
shown above, with the exception of selecting the FileSet, by using command
line arguments with a single command by entering:

\footnotesize
\begin{verbatim}
restore client=Rufus select current all done yes
\end{verbatim}
\normalsize

The {\bf client=Rufus} specification will automatically select Rufus as the
client, the {\bf current} tells Bareos that you want to restore the system to
the most current state possible, and the {\bf yes} suppresses the final {\bf
yes/mod/no} prompt and simply runs the restore.

The full list of possible command line arguments are:

\begin{itemize}
\item {\bf all} -- select all Files to be restored.
\item {\bf select} -- use the tree selection method.
\item {\bf done} -- do not prompt the user in tree mode.
\item {\bf copies} -- instead of using the actual backup jobs for restoring
   use the copies which were made of these backup Jobs. This could mean that
   on restore the client will contact a remote storage daemon if the data is
   copied to a remote storage daemon as part of your copy Job scheme.
\item {\bf current} -- automatically select the most current set of  backups
   for the specified client.
\item {\bf client=xxxx} -- initially specifies the client from which the
   backup was made and the client to which the restore will be make.  See also
   "restoreclient" keyword.
\item {\bf restoreclient=xxxx} -- if the keyword is specified, then the
   restore is written to that client.
\item {\bf jobid=nnn} -- specify a JobId or comma separated list of  JobIds to
   be restored.
\item {\bf before=YYYY-MM-DD HH:MM:SS} -- specify a date and time to  which
   the system should be restored. Only Jobs started before  the specified
   date/time will be selected, and as is the case  for {\bf current} Bareos will
   automatically find the most  recent prior Full save and all Differential and
   Incremental  saves run before the date you specify. Note, this command is  not
   too user friendly in that you must specify the date/time  exactly as shown.
\item {\bf file=filename} -- specify a filename to be restored. You  must
   specify the full path and filename. Prefixing the entry  with a less-than
   sign
   ({\textless}) will cause Bareos to assume that the  filename is on your system and
   contains a list of files to be  restored. Bareos will thus read the list from
   that file. Multiple  file=xxx specifications may be specified on the command
   line.
\item {\bf jobid=nnn} -- specify a JobId to be restored.
\item {\bf pool=pool-name} -- specify a Pool name to be used for selection  of
   Volumes when specifying options 5 and 6 (restore current system,  and restore
   current system before given date). This permits you to  have several Pools,
   possibly one offsite, and to select the Pool to  be used for restoring.
\item {\bf where=/tmp/bareos-restore} -- restore files in {\bf where} directory.
\item {\bf yes} -- automatically run the restore without prompting  for
   modifications (most useful in batch scripts).
\item {\bf strip\_prefix=/prod} -- remove a part of the filename when restoring.
\item {\bf add\_prefix=/test} -- add a prefix to all files when restoring (like
  where) (can't be used with {\bf where=}).
\item {\bf add\_suffix=.old} -- add a suffix to all your files.
\item {\bf regexwhere=!a.pdf!a.bkp.pdf!} -- do complex filename manipulation
  like with sed unix command. Will overwrite other filename manipulation. 
  For details, see the \ilink{regexwhere}{regexwhere} section.
\item {\bf restorejob=jobname} -- Pre-chooses a restore job. Bareos can be
  configured with multiple restore jobs ("Type = Restore" in the job
  definition). This allows the specification of different restore properties,
  including a set of RunScripts. When more than one job of this type is
  configured, during restore, Bareos will ask for a user selection
  interactively, or use the given restorejob.
\end{itemize}

\section{Using File Relocation}
\index[general]{File Relocation!using}
\label{filerelocation}
\label{restorefilerelocation}

\subsection*{Introduction}

The \textbf{where=} option is simple, but not very powerful. With file
relocation, Bareos can restore a file to the same directory, but with a
different name, or in an other directory without recreating the full path.

You can also do filename and path manipulations,
such as adding a suffix to all your files, renaming files
or directories, etc.  Theses options will overwrite {\bf where=} option.


For example, many users use OS snapshot features so that file
\texttt{/home/eric/mbox} will be backed up from the directory
\texttt{/.snap/home/eric/mbox}, which can complicate restores.  If you use
\textbf{where=/tmp}, the file will be restored to
\texttt{/tmp/.snap/home/eric/mbox} and you will have to move the file to
\texttt{/home/eric/mbox.bkp} by hand.

However, case, you could use the
\textbf{strip\_prefix=/.snap} and \textbf{add\_suffix=.bkp} options and
Bareos will restore the file to its original location -- that is
\texttt{/home/eric/mbox}.

To use this feature, there are command line options as described in
the \ilink{restore section}{restorefilerelocation} of this manual;
you can modify your restore job before running it; or you can
add options to your restore job in as described in
\linkResourceDirective{Dir}{Job}{Strip Prefix} and \linkResourceDirective{Dir}{Job}{Add Prefix}.

\begin{verbatim}
Parameters to modify:
     1: Level
     2: Storage
    ...
    10: File Relocation
    ...
Select parameter to modify (1-12):


This will replace your current Where value
     1: Strip prefix
     2: Add prefix
     3: Add file suffix
     4: Enter a regexp
     5: Test filename manipulation
     6: Use this ?
Select parameter to modify (1-6):
\end{verbatim}


\subsection*{RegexWhere Format}
    \label{regexwhere}

The format is very close to that used by sed or Perl (\texttt{s/replace this/by
  that/}) operator. A valid regexwhere expression has three fields :
\begin{itemize}
\item a search expression (with optional submatch)
\item a replacement expression (with optionnal back references \$1 to \$9)
\item a set of search options (only case-insensitive ``i'' at this time)
\end{itemize}

Each field is delimited by a separator specified by the user as the first
character of the expression. The separator can be one of the following:
\begin{verbatim}
<separator-keyword> = / ! ; % : , ~ # = &
\end{verbatim}

You can use several expressions separated by a commas.

\subsubsection*{Examples}

\begin{tabular}{|c|c|c|l|}
\hline
Orignal filename & New filename & RegexWhere & Comments \\
\hline
\hline
\texttt{c:/system.ini} & \texttt{c:/system.old.ini} & \texttt{/.ini\$/.old.ini/} & \$ matches end of name\\
\hline
\texttt{/prod/u01/pdata/} & \texttt{/rect/u01/rdata}  & \texttt{/prod/rect/,/pdata/rdata/} & uses two regexp\\
\hline
\texttt{/prod/u01/pdata/} & \texttt{/rect/u01/rdata}  & \texttt{!/prod/!/rect/!,/pdata/rdata/} & use \texttt{!} as separator\\
\hline
\texttt{C:/WINNT} & \texttt{d:/WINNT}  & \texttt{/c:/d:/i} & case insensitive pattern match \\
\hline

\end{tabular}

%\subsubsection{Using group}
%
%Like with Perl or Sed, you can make submatch with \texttt{()},
%
%\subsubsection*{Examples}


%\subsubsection{Options}
%
%       i   Do case-insensitive pattern matching.

\section{Restoring Directory Attributes}
\index[general]{Attributes!Restoring Directory}
\index[general]{Restoring Directory Attributes}

Depending how you do the restore, you may or may not get the directory entries
back to their original state. Here are a few of the problems you can
encounter, and for same machine restores, how to avoid them.

\begin{itemize}
\item You backed up on one machine and are restoring to another that is
   either a different OS or doesn't have the same users/groups defined.  Bareos
   does the best it can in these situations. Note, Bareos has saved the
   user/groups in numeric form, which means on a different machine, they
   may map to different user/group names.

\item You are restoring into a directory that is already created and has
   file creation restrictions.  Bareos tries to reset everything but
   without walking up the full chain of directories and modifying them all
   during the restore, which Bareos does and will not do, getting
   permissions back correctly in this situation depends to a large extent
   on your OS.

\item You are doing a recursive restore of a directory tree.  In this case
   Bareos will restore a file before restoring the file's parent directory
   entry.  In the process of restoring the file Bareos will create the
   parent directory with open permissions and ownership of the file being
   restored.  Then when Bareos tries to restore the parent directory Bareos
   sees that it already exists (Similar to the previous situation).  If you
   had set the Restore job's "Replace" property to "never" then Bareos will
   not change the directory's permissions and ownerships to match what it
   backed up, you should also notice that the actual number of files
   restored is less then the expected number.  If you had set the Restore
   job's "Replace" property to "always" then Bareos will change the
   Directory's ownership and permissions to match what it backed up, also
   the actual number of files restored should be equal to the expected
   number.

\item You selected one or more files in a directory, but did not select the
   directory entry to be restored.  In that case, if the directory is not
   on disk Bareos simply creates the directory with some default attributes
   which may not be the same as the original.  If you do not select a
   directory and all its contents to be restored, you can still select
   items within the directory to be restored by individually marking those
   files, but in that case, you should individually use the "markdir"
   command to select all higher level directory entries (one at a time) to
   be restored if you want the directory entries properly restored.

\end{itemize}

\section{Restoring on Windows}
\label{sec:RestoreOnWindows}
\index[general]{Restoring on Windows}
\index[general]{Windows!Restoring on}

If you are restoring on Windows systems, Bareos will restore the files
with the original ownerships and permissions as would be expected.  This is
also true if you are restoring those files to an alternate directory (using
the Where option in restore).  However, if the alternate directory does not
already exist, the Bareos File daemon (Client) will try to create it.  In
some cases, it may not create the directories, and if it does since the
File daemon runs under the SYSTEM account, the directory will be created
with SYSTEM ownership and permissions.  In this case, you may have problems
accessing the newly restored files.

To avoid this problem, you should create any alternate directory before
doing the restore.  Bareos will not change the ownership and permissions of
the directory if it is already created as long as it is not one of the
directories being restored (i.e.  written to tape).

The default restore location is {\bf /tmp/bareos-restores/} and if you are
restoring from drive {\bf E:}, the default will be
{\bf /tmp/bareos-restores/e/}, so you should ensure that this directory
exists before doing the restore, or use the {\bf mod} option to
select a different {\bf where} directory that does exist.

Some users have experienced problems restoring files that participate in
the Active Directory. They also report that changing the userid under which
Bareos (bareos-fd.exe) runs, from SYSTEM to a Domain Admin userid, resolves
the problem.




\section{Restore Errors}
\index[general]{Errors!Restore}
\index[general]{Restore Errors}

There are a number of reasons why there may be restore errors or
warning messages. Some of the more common ones are:

\begin{description}

\item [file count mismatch]
  This can occur for the following reasons:
  \begin{itemize}
  \item You requested Bareos not to overwrite existing or newer
     files.
  \item A Bareos miscount of files/directories. This is an
     on-going problem due to the complications of directories,
     soft/hard link, and such.  Simply check that all the files you
     wanted were actually restored.
  \end{itemize}

\item [file size error]
   When Bareos restores files, it checks that the size of the
   restored file is the same as the file status data it saved
   when starting the backup of the file. If the sizes do not
   agree, Bareos will print an error message. This size mismatch
   most often occurs because the file was being written as Bareos
   backed up the file. In this case, the size that Bareos
   restored will be greater than the status size.  This often
   happens with log files.

   If the restored size is smaller, then you should be concerned
   about a possible tape error and check the Bareos output as
   well as your system logs.
\end{description}



\section{Example Restore Job Resource}
\index[general]{Resource!Example Restore Job}

\footnotesize
\begin{verbatim}
Job {
  Name = "RestoreFiles"
  Type = Restore
  Client = Any-client
  FileSet = "Any-FileSet"
  Storage = Any-storage
  Where = /tmp/bareos-restores
  Messages = Standard
  Pool = Default
}
\end{verbatim}
\normalsize

If {\bf Where} is not specified, the default location for restoring files will
be their original locations.
\label{Selection}

\section{File Selection Commands}
\index[general]{Console!File Selection}
\index[general]{File Selection Commands}

After you have selected the Jobs to be restored and Bareos has created the
in-memory directory tree, you will enter file selection mode as indicated by
the dollar sign ({\bf \$}) prompt. While in this mode, you may use the
commands listed above. The basic idea is to move up and down the in memory
directory structure with the {\bf cd} command much as you normally do on the
system. Once you are in a directory, you may select the files that you want
restored. As a default no files are marked to be restored. If you wish to
start with all files, simply enter: {\bf cd /} and {\bf mark *}. Otherwise
proceed to select the files you wish to restore by marking them with the {\bf
mark} command. The available commands are:

\begin{description}

\item [cd]
   \index[general]{Console!File Selection!cd}
  The {\bf cd} command changes the current directory to the argument specified.
  It operates much like the Unix {\bf cd} command.  Wildcard specifications are
  not permitted.

  Note, on Windows systems, the various drives (c:, d:, ...) are treated like a
  directory within the file tree while in the file selection mode. As a
  consequence, you must do a {\bf cd c:} or possibly in some cases a {\bf cd
    C:} (note upper case) to get down to the first directory.

\item [dir]
   \index[general]{Console!File Selection!dir}
   The {\bf dir} command is similar to the {\bf ls} command, except that it
   prints it in long format (all details). This command can be a bit slower
   than the {\bf ls} command because it must access the catalog database for
   the detailed information for each file.

\item [estimate]
   \index[general]{Console!File Selection!estimate}
   The {\bf estimate} command prints a summary of the total files in the tree,
   how many are marked to be restored, and an estimate of the number of bytes
   to be restored. This can be useful if you are short on disk space on the
   machine where the files will be restored.

\item [find]
   \index[general]{Console!File Selection!find}
   The {\bf find} command accepts one or more arguments  and displays all files
   in the tree that match that argument. The argument  may have wildcards. It is
   somewhat similar to the Unix command  {\bf find / -name arg}.

\item [ls]
   \index[general]{Console!File Selection!ls}
   The {\bf ls} command produces a listing of all the files  contained in the
   current directory much like the Unix {\bf ls} command.  You may specify an
   argument containing wildcards, in which case only  those files will be
   listed.

   Any file that is marked to be restored will  have its name preceded by an
   asterisk ({\bf *}). Directory names  will be terminated with a forward slash
   ({\bf /}) to distinguish them  from filenames.

\item [lsmark]
   \index[general]{Console!File Selection!lsmark}
   The {\bf lsmark} command is the same as the  {\bf ls} except that it will
   print only those files marked for  extraction. The other distinction is that
   it will recursively  descend into any directory selected.

\item [mark]
   \index[general]{Console!File Selection!mark}
   The {\bf mark} command allows you to mark files to be restored. It takes a
   single argument which is the filename  or directory name in the current
   directory to be marked for extraction.  The argument may be a wildcard
   specification, in which  case all files that match in the current directory
   are marked to be  restored. If the argument matches a directory rather than a
   file,  then the directory and all the files contained in that directory
   (recursively)  are marked to be restored. Any marked file will have its name
   preceded with an asterisk ({\bf *}) in the output produced by the  {\bf ls}
or
   {\bf dir} commands. Note, supplying a full path on  the mark command does not
   work as expected to select a file or  directory in the current directory.
   Also, the {\bf mark} command works  on the current and lower directories but
   does not touch higher level  directories.

   After executing the {\bf mark} command, it will print a brief summary:

\footnotesize
\begin{verbatim}
    No files marked.

\end{verbatim}
\normalsize

   If no files were marked, or:

\footnotesize
\begin{verbatim}
    nn files marked.

\end{verbatim}
\normalsize

   if some files are marked.

\item [unmark]
   \index[general]{Console!File Selection!unmark}
   The {\bf unmark} is identical to the {\bf mark}  command, except that it
   unmarks the specified file or files so that  they will not be restored. Note:
   the {\bf unmark} command works from  the current directory, so it does not
   unmark any files at a higher  level. First do a {\bf cd /} before the {\bf
   unmark *} command if  you want to unmark everything.

\item [pwd]
   \index[general]{Console!File Selection!pwd}
   The {\bf pwd} command prints the current working  directory. It accepts no
   arguments.

\item [count]
   \index[general]{Console!File Selection!count}
   The {\bf count} command prints the total files in the  directory tree and the
   number of files marked to be restored.

\item [done]
   \index[general]{Console!File Selection!done}
   This command terminates file selection mode.

\item [exit]
   \index[general]{Console!File Selection!exit}
   This command terminates file selection mode (the same as  done).

\item [quit]
   \index[general]{Console!File Selection!quit}
   This command terminates the file selection and does  not run the restore
job.


\item [help]
   \index[general]{Console!File Selection!help}
   This command prints a summary of the commands available.

\item [?]
   \index[general]{Console!File Selection!?}
   This command is the same as the {\bf help} command.
\end{description}

If your filename contains some weird caracters, you can use \texttt{?},
\texttt{*} or \textbackslash{}\textbackslash{}. For example, if your filename
contains a \textbackslash{}, you can use
\textbackslash{}\textbackslash{}\textbackslash{}\textbackslash{}.

\begin{verbatim}
* mark weird_file\\\\with-backslash
\end{verbatim}

