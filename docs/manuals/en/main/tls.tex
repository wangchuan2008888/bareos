
\chapter{Transport Encryption}
\label{CommEncryption}
\label{sec:TransportEncryption}
\index[general]{Communications Encryption}
\index[general]{Encryption!Communication}
\index[general]{Encryption!Transport}
\index[general]{Transport Encryption}
\index[general]{TLS}
\index[general]{SSL}

Bareos TLS (Transport Layer Security) is built-in network
encryption code to provide secure network transport similar to
that offered by \command{stunnel} or \command{ssh}.  The data written to
Volumes by the Storage daemon is not encrypted by this code.
For data encryption, please see the \nameref{DataEncryption} chapter.

The initial Bacula encryption implementation has been written by Landon Fuller.

Supported features of this code include:
\begin{itemize}
\item Client/Server TLS Requirement Negotiation
\item TLSv1 Connections with Server and Client Certificate
Validation
\item Forward Secrecy Support via Diffie-Hellman Ephemeral Keying
\end{itemize}

This document will refer to both \bquote{server} and \bquote{client} contexts.
These terms refer to the accepting and initiating peer, respectively.

Diffie-Hellman anonymous ciphers are not supported by this code.  The
use of DH anonymous ciphers increases the code complexity and places
explicit trust upon the two-way CRAM-MD5 implementation.  CRAM-MD5 is
subject to known plaintext attacks, and it should be considered
considerably less secure than PKI certificate-based authentication.

\section{TLS Configuration Directives}
\label{TlsDirectives}
Additional configuration directives have been added to all the daemons
(Director, File daemon, and Storage daemon) as well as the various
different Console programs.
These directives are defined as follows:

\begin{description}
\directive{dir}{TLS Enable}{yes{\textbar}no}{}{no}{}%
Enable TLS support. Without setting \configdirective{TLS Require}=yes,
the connection can fall back to unencrypted connection,
if the other side does not support TLS.

\directive{dir}{TLS Require}{yes{\textbar}no}{}{no}{}%
Require TLS connections.
If TLS is not required,
then Bareos will connect with other daemons either with or without TLS depending
on what the other daemon requests.
If TLS is required,
then Bareos will refuse any connection that does not use TLS.
\configdirective{TLS Require}=yes  implicitly sets \configdirective{TLS Enable}=yes.

\directive{dir}{TLS Certificate}{filename}{}{}{}%
The full path and filename of a PEM encoded TLS certificate.  It can be
used as either a client or server certificate.
It is used because PEM files are base64 encoded and hence ASCII
text based rather than binary.
They may also contain encrypted information.

\directive{dir}{TLS Key}{filename}{}{}{}%
The full path and filename of a PEM encoded TLS private key.  It must
correspond to the certificate specified in the \configdirective{TLS Certificate} configuration directive.

\directive{dir}{TLS Verify Peer}{yes{\textbar}no}{}{}{}%
Request and verify the peers certificate.

In server context, unless the \configdirective{TLS Allowed CN} configuration directive is specified,
any client certificate signed by a known-CA will be accepted.

In client context, the server certificate CommonName attribute is checked against
the \configdirective{Address} and \configdirective{TLS Allowed CN} configuration directives.


\directive{dir}{TLS Allowed CN}{stringlist}{}{}{}%
Common name attribute of allowed peer certificates.
If \configdirective{TLS Verify Peer}=yes, all connection request certificates
will be checked against this list.

This directive may be specified more than once.


\directive{dir}{TLS CA Certificate File}{filename}{}{}{}%
The full path and filename specifying a
PEM encoded TLS CA certificate(s).  Multiple certificates are
permitted in the file.

In a client context, one of
\configdirective{TLS CA Certificate File} or \configdirective{TLS CA Certificate Dir}
is required.

In a server context, it is only required if \configdirective{TLS Verify Peer} is used.

\directive{dir}{TLS CA Certificate Dir}{directory}{}{}{}%
Full path to TLS CA certificate directory.  In the current implementation,
certificates must be stored PEM encoded with OpenSSL-compatible hashes,
which is the subject name's hash and an extension of {\bf .0}.

In a client context, one of
\configdirective{TLS CA Certificate File} or \configdirective{TLS CA Certificate Dir}
is required.

In a server context, it is only required if \configdirective{TLS Verify Peer} is used.


\directive{dir}{TLS DH File}{filename}{}{}{}%
Path to PEM encoded Diffie-Hellman parameter file.  If this directive is
specified, DH key exchange will be used for the ephemeral keying, allowing
for forward secrecy of communications.  DH key exchange adds an additional
level of security because the key used for encryption/decryption by the
server and the client is computed on each end and thus is never passed over
the network if Diffie-Hellman key exchange is used.  Even if DH key
exchange is not used, the encryption/decryption key is always passed
encrypted.  This directive is only valid within a server context.

To generate the parameter file, you
may use openssl:

\begin{commands}{create DH key}
openssl dhparam -out dh1024.pem -5 1024
\end{commands}

\end{description}

\section{Getting TLS Certificates}

To get a trusted certificate (CA or Certificate Authority signed
certificate), you will either need to purchase certificates signed by a
commercial CA or become a CA
yourself, and thus you can sign all your own certificates.

Bareos is known to work well with RSA certificates.

You can use programs like \elink{xca}{http://xca.sourceforge.net/} or TinyCA
to easily manage your own CA with a Graphical User Interface.



\section{Example TLS Configuration Files}
\index[general]{Example!TLS Configuration Files}
\index[general]{TLS Configuration Files}

An example of the TLS portions of the configuration
files are listed below.

Another example can be found at \bareosTlsConfigurationExample.

\subsection{Bareos Director}

\begin{bareosConfigResource}{bareos-dir}{director}{bareos-dir}
Director {                            # define myself
    Name = bareos-dir
    ...
    TLS Enable = yes
    TLS Require = yes
    TLS CA Certificate File = /etc/bareos/tls/ca.pem
    # This is a server certificate, used for incoming
    # (console) connections.
    TLS Certificate = /etc/bareos/tls/bareos-dir.example.com-cert.pem
    TLS Key = /etc/bareos/tls/bareos-dir.example.com-key.pem
    TLS Verify Peer = yes
    TLS Allowed CN = "bareos@backup1.example.com"
    TLS Allowed CN = "administrator@example.com"
}
\end{bareosConfigResource}

\begin{bareosConfigResource}{bareos-dir}{storage}{File}
Storage {
    Name = File
    Address = bareos-sd1.example.com
    ...
    TLS Require = yes
    TLS CA Certificate File = /etc/bareos/tls/ca.pem
    # This is a client certificate, used by the director to
    # connect to the storage daemon
    TLS Certificate = /etc/bareos/tls/bareos-dir.example.com-cert.pem
    TLS Key = /etc/bareos/tls/bareos-dir.example.com-key.pem
    TLS Allowed CN = bareos-sd1.example.com
}
\end{bareosConfigResource}

\begin{bareosConfigResource}{bareos-dir}{client}{client1-fd}
Client {
    Name = client1-fd
    Address = client1.example.com
    ...
    TLS Enable = yes
    TLS Require = yes
    TLS CA Certificate File = /etc/bareos/tls/ca.pem
    TLS Certificate = "/etc/bareos/tls/bareos-dir.example.com-cert.pem"
    TLS Key = "/etc/bareos/tls/bareos-dir.example.com-key.pem"
    TLS Allowed CN = client1.example.com
}
\end{bareosConfigResource}



\subsection{Bareos Storage Daemon}

\begin{bareosConfigResource}{bareos-sd}{storage}{bareos-sd1}
Storage {
    Name = bareos-sd1
    ...
    # These TLS configuration options are used for incoming
    # file daemon connections. Director TLS settings are handled
    # in Director resources.
    TLS Enable = yes
    TLS Require = yes
    TLS CA Certificate File = /etc/bareos/tls/ca.pem
    # This is a server certificate. It is used by connecting
    # file daemons to verify the authenticity of this storage daemon
    TLS Certificate = /etc/bareos/tls/bareos-sd1.example.com-cert.pem
    TLS Key = /etc/bareos/tls/bareos-sd1.example.com-key.pem
    # Peer verification must be disabled,
    # or all file daemon CNs must be listed in "TLS Allowed CN".
    # Peer validity is verified by the storage connection cookie
    # provided to the File Daemon by the Director.
    TLS Verify Peer = no
}
\end{bareosConfigResource}

\begin{bareosConfigResource}{bareos-sd}{director}{bareos-dir}
Director {
    Name = bareos-dir
    ...
    TLS Enable = yes
    TLS Require = yes
    TLS CA Certificate File = /etc/bareos/tls/ca.pem
    # This is a server certificate. It is used by the connecting
    # director to verify the authenticity of this storage daemon
    TLS Certificate = /etc/bareos/tls/bareos-sd1.example.com-cert.pem
    TLS Key = /etc/bareos/tls/bareos-sd1.example.com-key.pem
    # Require the connecting director to provide a certificate
    # with the matching CN.
    TLS Verify Peer = yes
    TLS Allowed CN = "bareos-dir.example.com"
}
\end{bareosConfigResource}



\subsection{Bareos File Daemon}

\begin{bareosConfigResource}{bareos-fd}{client}{myself}
Client {
    Name = client1-fd
    ...
    # you need these TLS entries so the SD and FD can
    # communicate
    TLS Enable = yes
    TLS Require = yes

    TLS CA Certificate File = /etc/bareos/tls/ca.pem
    TLS Certificate = /etc/bareos/tls/client1.example.com-cert.pem
    TLS Key = /etc/bareos/tls/client1.example.com-key.pem

    TLS Allowed CN = bareos-sd1.example.com
}
\end{bareosConfigResource}

\begin{bareosConfigResource}{bareos-fd}{director}{bareos-dir}
Director {
    Name = bareos-dir
    ...
    TLS Enable = yes
    TLS Require = yes
    TLS CA Certificate File = /etc/bareos/tls/ca.pem
    # This is a server certificate. It is used by connecting
    # directors to verify the authenticity of this file daemon
    TLS Certificate = /etc/bareos/tls/client11.example.com-cert.pem
    TLS Key = /etc/bareos/tls/client1.example.com-key.pem
    TLS Verify Peer = yes
    # Allow only the Director to connect
    TLS Allowed CN = "bareos-dir.example.com"
}
\end{bareosConfigResource}
