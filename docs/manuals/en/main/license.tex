\chapter{Bareos Copyright, Trademark, and Licenses}
\label{LicenseChapter}
\index[general]{License!Bareos Copyright Trademark Licenses}
\index[general]{Bareos Copyright, Trademark, and Licenses}

\section{Licenses Overview}

There are a number of different licenses that are used in Bareos.

\subsection*{FDL}
\index[general]{License!FDL}

The \ilink{GNU Free Documentation License (FDL)}{fdl} is used for this manual,
which is a free and open license. This means that you may freely
reproduce it and even make changes to it.


\subsection*{AGPL}
\index[general]{License!AGPL}

The vast bulk of the source code is released under the
\ilink{GNU Affero General Public License (AGPL) version 3}{agpl}.

Most of this code is copyrighted: Copyright \copyright 2000-2012
Free Software Foundation Europe e.V.

All new code is copyrighted: Copyright \copyright 2013-2013 Bareos GmbH \& Co. KG

Portions may be copyrighted by other people.  These files are released
under different licenses which are compatible with the AGPL license.

\subsection*{LGPL}
\index[general]{License!LGPL}

Some of the Bareos library source code is released under the
\ilink{GNU Lesser General Public License (LGPL)}{lgpl}. This
permits third parties to use these parts of our code in their proprietary
programs to interface to Bareos.

\subsection*{Public Domain}
\index[general]{License!Public Domain}

Some of the Bareos code, or code that Bareos references, has been released
to the public domain. E.g. md5.c, SQLite.

\subsection*{Trademark}
\index[general]{Trademark}

Bareos\raisebox{.6ex}{\textsuperscript{\textregistered}} is a registered
trademark of Bareos GmbH \& Co. KG.

Bacula\raisebox{.6ex}{\textsuperscript{\textregistered}} is a registered
trademark of Kern Sibbald.

\subsection*{Disclaimer}
\index[general]{Disclaimer}

NO WARRANTY

BECAUSE THE PROGRAM IS LICENSED FREE OF CHARGE, THERE IS NO WARRANTY FOR THE
PROGRAM, TO THE EXTENT PERMITTED BY APPLICABLE LAW. EXCEPT WHEN OTHERWISE
STATED IN WRITING THE COPYRIGHT HOLDERS AND/OR OTHER PARTIES PROVIDE THE
PROGRAM "AS IS" WITHOUT WARRANTY OF ANY KIND, EITHER EXPRESSED OR IMPLIED,
INCLUDING, BUT NOT LIMITED TO, THE IMPLIED WARRANTIES OF MERCHANTABILITY AND
FITNESS FOR A PARTICULAR PURPOSE. THE ENTIRE RISK AS TO THE QUALITY AND
PERFORMANCE OF THE PROGRAM IS WITH YOU. SHOULD THE PROGRAM PROVE DEFECTIVE,
YOU ASSUME THE COST OF ALL NECESSARY SERVICING, REPAIR OR CORRECTION.

IN NO EVENT UNLESS REQUIRED BY APPLICABLE LAW OR AGREED TO IN WRITING WILL ANY
COPYRIGHT HOLDER, OR ANY OTHER PARTY WHO MAY MODIFY AND/OR REDISTRIBUTE THE
PROGRAM AS PERMITTED ABOVE, BE LIABLE TO YOU FOR DAMAGES, INCLUDING ANY
GENERAL, SPECIAL, INCIDENTAL OR CONSEQUENTIAL DAMAGES ARISING OUT OF THE USE
OR INABILITY TO USE THE PROGRAM (INCLUDING BUT NOT LIMITED TO LOSS OF DATA OR
DATA BEING RENDERED INACCURATE OR LOSSES SUSTAINED BY YOU OR THIRD PARTIES OR
A FAILURE OF THE PROGRAM TO OPERATE WITH ANY OTHER PROGRAMS), EVEN IF SUCH
HOLDER OR OTHER PARTY HAS BEEN ADVISED OF THE POSSIBILITY OF SUCH DAMAGES.

\subsection*{Other Copyrights and Trademarks}

Certain words and/or products are Copyrighted or Trademarked such as Windows (by Microsoft). Since
they are numerous, and we are not necessarily aware of the details of each, we don’t try to list them here.
However, we acknowledge all such Copyrights and Trademarks, and if any copyright or trademark holder
wishes a specific acknowledgment, notify us, and we will be happy to add it where appropriate.


    %\include{fdl}
\pagebreak
\section{GNU Free Documentation License}
    \label{fdl}
    \verbatiminput{fdl.txt}

    %\include{agpl}
\pagebreak
\section{GNU Affero Gerneral Public License}
    \label{agpl}
    \verbatiminput{agpl.txt}

    %\include{lesser}
\pagebreak
\section{GNU Lesser Gerneral Public License}
    \label{lgpl}
    \verbatiminput{lgpl.txt}
