\defDirective{Dir}{Storage}{Address}{}{}{%
Where the address is a host name,  a {\bf fully qualified domain name}, or an
{\bf IP address}. Please note  that the {\textless}address{\textgreater} as specified here
will be transmitted to  the File daemon who will then use it to contact the
Storage daemon. Hence,  it is {\bf not}, a good idea to use {\bf localhost} as
the  name but rather a fully qualified machine name or an IP address.  This
directive is required.
}

\defDirective{Dir}{Storage}{Allow Compression}{}{}{%
This directive is optional, and if you specify {\bf No},
it will cause backups jobs running on this storage resource to run
without client File Daemon compression.  This effectively overrides
compression options in FileSets used by jobs which use this storage
resource.
\label{AllowCompression}
}

\defDirective{Dir}{Storage}{Auth Type}{}{}{%
Specifies the authentication type that must be supplied when connecting to
a backup protocol that uses a specific authentication type.
}

\defDirective{Dir}{Storage}{Auto Changer}{}{}{%
When \linkResourceDirective{Dir}{Storage}{Device} refers to an Auto Changer (\linkResourceDirective{Sd}{Device}{Autochanger}),
this directive must be set to \parameter{yes}.

If you specify \parameter{yes},
\begin{itemize}
  \item Volume management command like \bcommand{label}{} or \bcommand{add}{} will request a Autochanger Slot number.
  \item Bareos will prefer Volumes, that are in a Auto Changer slot.
    If none of theses volumes can be used, even after recycling, pruning, ...,
    Bareos will search for any volume of the same \linkResourceDirective{Dir}{Storage}{Media Type} whether or not in the magazine.
\end{itemize}

Please consult the \nameref{AutochangersChapter} chapter for details.
}

\defDirective{Dir}{Storage}{Collect Statistics}{}{}{%
Collect statistic information. These information will be collected by the Director (see \linkResourceDirective{Dir}{Director}{Statistics Collect Interval}) and stored in the Catalog.
}

\defDirective{Dir}{Storage}{Description}{}{}{%
Information.
}

\defDirective{Dir}{Storage}{Device}{}{}{%

If \linkResourceDirective{Dir}{Job}{Protocol} is not \parameter{NDMP_NATIVE} (default is \linkResourceDirectiveValue{Dir}{Job}{Protocol}{Native}), this directive refers to one or multiple \linkResourceDirective{Sd}{Device}{Name}
or a single \linkResourceDirective{Sd}{Autochanger}{Name}.

If an Autochanger should be used, it had to refer to a configured \linkResourceDirective{Sd}{Autochanger}{Name}.
In this case, also set \linkResourceDirectiveValue{Dir}{Storage}{Auto Changer}{yes}.

Otherwise it refers to one or more configured \linkResourceDirective{Sd}{Device}{Name}, see  \nameref{sec:MultipleStorageDevices}.

This name is not the physical device name, but the logical device name as
defined in the \bareosSd resource.

If \resourceDirectiveValue{Dir}{Job}{Protocol}{NDMP_NATIVE}, it refers to tape devices on the NDMP \TapeAgent, see \nameref{sec:NdmpNative}.
}



\defDirective{Dir}{Storage}{Enabled}{}{}{%
}

\defDirective{Dir}{Storage}{Heartbeat Interval}{}{}{%
This directive is optional and if specified will cause the Director to
set a keepalive interval (heartbeat) in seconds on each of the sockets
it opens for the Storage resource.  This value will override any
specified at the Director level.  It is implemented only on systems
(Linux, ...) that provide the {\bf setsockopt} TCP\_KEEPIDLE function.
The default value is zero, which means no change is made to the socket.
}


\defDirective{Dir}{Storage}{Lan Address}{}{}{%
This directive might be useful in network setups where the \bareosDir and \bareosFd need different addresses to communicate with the \bareosSd.

For details, see \nameref{LanAddress}.

This directive corresponds to \linkResourceDirective{Dir}{Client}{Lan Address}.
}

\defDirective{Dir}{Storage}{Maximum Bandwidth Per Job}{}{}{%
}

\defDirective{Dir}{Storage}{Maximum Concurrent Jobs}{}{}{%
This directive specifies the maximum number of Jobs with the current
Storage resource that can run concurrently.  Note, this directive limits
only Jobs for Jobs using this Storage daemon.  Any other restrictions on
the maximum concurrent jobs such as in the Director, Job or Client
resources will also apply in addition to any limit specified here.

If you set the Storage daemon's number of concurrent jobs greater than one,
we recommend that you read \nameref{ConcurrentJobs} and/or
turn data spooling on as documented in \nameref{SpoolingChapter}.
}

\defDirective{Dir}{Storage}{Maximum Concurrent Read Jobs}{}{}{%
This directive specifies the maximum number of Jobs with the current
Storage resource that can read concurrently.
}

\defDirective{Dir}{Storage}{Media Type}{}{}{%
This directive specifies the Media Type to be used to store the data.
This is an arbitrary string of characters up to 127 maximum that you
define.  It can be anything you want.  However, it is best to make it
descriptive of the storage media (e.g.  File, DAT, "HP DLT8000", 8mm,
...).  In addition, it is essential that you make the {\bf Media Type}
specification unique for each storage media type.  If you have two DDS-4
drives that have incompatible formats, or if you have a DDS-4 drive and
a DDS-4 autochanger, you almost certainly should specify different {\bf
Media Types}.  During a restore, assuming a {\bf DDS-4} Media Type is
associated with the Job, Bareos can decide to use any Storage daemon
that supports Media Type {\bf DDS-4} and on any drive that supports it.

If you are writing to disk Volumes, you must make doubly sure that each
Device resource defined in the Storage daemon (and hence in the
Director's conf file) has a unique media type.  Otherwise Bareos
may assume, these Volumes can be mounted and read by any Storage daemon File device.

Currently Bareos permits only a single Media Type per Storage
Device definition. Consequently, if
you have a drive that supports more than one Media Type, you can
give a unique string to Volumes with different intrinsic Media
Type (Media Type = DDS-3-4 for DDS-3 and DDS-4 types), but then
those volumes will only be mounted on drives indicated with the
dual type (DDS-3-4).

If you want to tie Bareos to using a single Storage daemon or drive, you
must specify a unique Media Type for that drive.  This is an important
point that should be carefully understood.  Note, this applies equally
to Disk Volumes.  If you define more than one disk Device resource in
your Storage daemon's conf file, the Volumes on those two devices are in
fact incompatible because one can not be mounted on the other device
since they are found in different directories.  For this reason, you
probably should use two different Media Types for your two disk Devices
(even though you might think of them as both being File types).  You can
find more on this subject in the \ilink{Basic Volume
Management}{DiskChapter} chapter of this manual.

The {\bf MediaType} specified in the Director's Storage resource, {\bf
must} correspond to the {\bf Media Type} specified in the {\bf Device}
resource of the {\bf Storage daemon} configuration file.  This directive
is required, and it is used by the Director and the Storage daemon to
ensure that a Volume automatically selected from the Pool corresponds to
the physical device.  If a Storage daemon handles multiple devices (e.g.
will write to various file Volumes on different partitions), this
directive allows you to specify exactly which device.

As mentioned above, the value specified in the Director's Storage
resource must agree with the value specified in the Device resource in
the {\bf Storage daemon's} configuration file.  It is also an additional
check so that you don't try to write data for a DLT onto an 8mm device.
\label{MediaType}
}

\defDirective{Dir}{Storage}{Name}{}{}{%
The name of the storage resource. This  name appears on the Storage directive
specified in the Job resource and is required.
}

\defDirective{Dir}{Storage}{Paired Storage}{}{}{%
For NDMP backups this points to the definition of the Native Storage
that is accesses via the NDMP protocol. For now we only support NDMP
backups and restores to access Native Storage Daemons via the NDMP
protocol. In the future we might allow to use Native NDMP storage which
is not bound to a Bareos Storage Daemon.
}

\defDirective{Dir}{Storage}{Password}{}{}{%
This is the password to be used  when establishing a connection with the
Storage services. This  same password also must appear in the Director
resource of the Storage  daemon's configuration file. This directive is
required.

The password is plain text.
}

\defDirective{Dir}{Storage}{Port}{}{}{%
Where port is the port to use to  contact the storage daemon for information
and to start jobs.  This same port number must appear in the Storage resource
of the  Storage daemon's configuration file.
}

\defDirective{Dir}{Storage}{Protocol}{}{}{%
}

\defDirective{Dir}{Storage}{SD Address}{}{}{%
}

\defDirective{Dir}{Storage}{SD Password}{}{}{%
}

\defDirective{Dir}{Storage}{SD Port}{}{}{%
}

\defDirective{Dir}{Storage}{Sdd Port}{}{}{%
}

\defDirective{Dir}{Storage}{TLS Authenticate}{}{}{%
}

\defDirective{Dir}{Storage}{TLS CA Certificate File}{}{}{%
}

\defDirective{Dir}{Storage}{TLS CA Certificate Dir}{}{}{%
}

\defDirective{Dir}{Storage}{TLS Certificate}{}{}{%
}

\defDirective{Dir}{Storage}{TLS Certificate Revocation List}{}{}{%
}

\defDirective{Dir}{Storage}{TLS Enable}{}{}{%
Bareos can be configured to encrypt all its network traffic.
For details, refer to chapter \nameref{TlsDirectives}.
}

\defDirective{Dir}{Storage}{TLS Key}{}{}{%
}

\defDirective{Dir}{Storage}{TLS Require}{}{}{%
}

\defDirective{Dir}{Storage}{Username}{}{}{%
}
